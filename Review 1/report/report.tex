\documentclass[12pt]{article}

\usepackage[margin=1in]{geometry}
\usepackage{amsfonts,amsmath,amssymb}
\usepackage{multicol}

\usepackage{graphicx}
\usepackage{float}
\usepackage[nottoc, notlot, notlof]{tocbibind}
\usepackage{hyperref}
\usepackage{enumitem}
\usepackage{caption}
\usepackage{subcaption}
\usepackage[T1]{fontenc}
\usepackage[utf8]{inputenc}
\usepackage{xcolor}
\newcommand{\leavealine}
{\vskip 0.5cm}


% for equation box
\usepackage{color}
\definecolor{myblue}{rgb}{.8, .8, 1}

\usepackage[most]{tcolorbox}
\tcbset{
	enhanced,
	colback=myblue!100!white,
	boxrule=0.1pt,
	colframe=myblue!100!black,
	fonttitle=\bfseries
}
%until here

% for python code
\usepackage[utf8]{inputenc}

% Default fixed font does not support bold face
\DeclareFixedFont{\ttb}{T1}{txtt}{bx}{n}{12} % for bold
\DeclareFixedFont{\ttm}{T1}{txtt}{m}{n}{12}  % for normal

% Custom colors
\definecolor{deepblue}{rgb}{0,0,0.5}
\definecolor{deepred}{rgb}{0.6,0,0}
\definecolor{deepgreen}{rgb}{0,0.5,0}

% \usepackage{listings}

% Python style for highlighting
\usepackage[utf8]{inputenc}

% Default fixed font does not support bold face
\DeclareFixedFont{\ttb}{T1}{txtt}{bx}{n}{12} % for bold
\DeclareFixedFont{\ttm}{T1}{txtt}{m}{n}{12}  % for normal

\DeclareSymbolFont{matha}{OML}{txmi}{m}{it}% txfonts
\DeclareMathSymbol{\varv}{\mathord}{matha}{118}
% Custom colors
\usepackage{color}
\definecolor{deepblue}{rgb}{0,0,0.5}
\definecolor{deepred}{rgb}{0.6,0,0}
\definecolor{deepgreen}{rgb}{0,0.5,0}
\definecolor{almostwhite}{rgb}{0.99,0.99, 0.99}

\usepackage{listings}

% Python style for highlighting
\lstset{
	basicstyle  =   \footnotesize,
	keywordstyle    = \color{deepred}\bfseries,
	stringstyle     = \color{strings},
	identifierstyle = \color{black},
	commentstyle    = \color{deepgreen},
	breaklines=true,
	numbersep=-10pt,
	stepnumber=1,
	showspaces=false,
	escapechar=§,
	showstringspaces=false,
	showtabs=false,
	frame=single,  
	rulecolor=\color{black},
	tabsize=4,              
	captionpos=t,           
	breaklines=true,
	breakatwhitespace=false,
	numbers=left,
	extendedchars=\true,
	emph=[3]{href, Particle, Boris_update, Field, uniform_E_field, radial_E_field, uniform_B_field, helmholtz_coil_B_field, two_helmholtz_B_field, Sampler, sample_uniform_position, sample_uniform_velocity, sample_velocity_uniformKE, sample_Maxwellian_velocity, sample_parabolic_velocity},
	emphstyle=[3]{\color{deepblue}},
	backgroundcolor=\color{almostwhite},
	language=Python 
}

%until here

\begin{document}
	
	\fontfamily{ppl}\selectfont 
		\begin{center}
			\large{\textbf{MEE4099 Capstone Project}} \hspace{1cm}\large{\textbf{Review 1 Report}} \\
			 \large{Project Title:} \\ \Large{\textbf{Magnetic Mirror Effect in Magnetron Plasma:}} \\
			\Large{\textbf{Modeling of Plasma Parameters}} \\
		\end{center}
		%\color{blue}
	\textbf{Project ID:} 21BTECH10051 \\
	\textbf{Team Members:} \\
	18BEM0145 Sashi Kant Shah \\
	18BME2104 Kaushal Timilsina \\
	18BME2109 Hrishav Mishra \\
	
	\noindent \textbf{Internal Guide:} \\
	Professor Sitaram Dash
	
	\section{Introduction}
	One of the team members- 18BME2104 Kaushal studied the class MEE4005 Surface Engineering taught by Professor Sitaram Dash- our internal guide during the Fall Semester 2021. Many interesting plasma based surface engineering techniques were studied during the class, one such technique being Magnetron Sputtering. This inspired the study of plasma in this project.
	
	\subsection{Plasma}
	One comes across many definitions of plasma including: Fourth state of matter, Ionized gas, etc. However, it is best to describe plasma with some characteristic parameters, when one attempts to describe a plasma quantitatively. Some quantities that help define a plasma are:
	\begin{enumerate}
		\item Number density, $n$ \\
		Number density of a plasma describes the number of particles per unit volume. Plasma contains charged particles or ionized species. However, a plasma might at the same time also contain neutral atoms and molecules but also particles of different species- charged and neutral. If multiple species are contained in a plasma system, number densities of each species could be used to describe the system. For example, a plasma may contain electrons, charged ions and neutral atoms and molecules. Mass density $\varrho$ is defined as $\varrho := m n$ and is often used alongside number density, where $m$ is the mass of the species.
		
		\item Ionization, $\alpha$ \\
		Defined as $\alpha := \frac{\displaystyle n_{charged}}{\displaystyle n_{charged} + n_{neutral}}$, the ionization of a plasma describes the fraction of charged particles, with $\alpha = 1$ meaning that all the particles are charged and $\alpha = 0$ meaning that all the particles are neutral.
		
		\item Temperature, $T$ \\
		The temperature of a plasma describes the average kinetic energy of the particles in the plasma. When a gas is ionized to form a plasma, the ionization $\alpha$ can depend on the temperature of the plasma.
		
		\item Mean free path, $\lambda_{mfp}$ \\
		The gas-like behavior of a plasma is characterized by mean free path of particles much larger than the scale of the plasma. The mean free path is influenced by the temperature of the plasma. The mean free path and the thermal velocity of the particles as described by the temperature, are related by the timescale of collisions as $\lambda := v_{th} \tau$ where $\tau$ is the timescale of collisions. 
		
		\item Debye Length, $\lambda_{D}$ \\
		In a plasma, electrostatic Coulomb interactions between charged particles compete with random thermal speed of the particles described by the temperature of the plasma. The Debye sphere is an imaginary sphere around a charged particle, where oppositely charged particles are attracted and in doing so screen the charge of the central particle to the outer plasma so that the electrostatic influence of a particle is limited to the Debye sphere surrounding it. This is why plasma's are often said to be Quasi-neutral as charge screening leads to a neutral behavior electrostatically on a scale much larger than the Debye length. The Debye length is defined as the radius of the Debye sphere.
		
		\item Plasma beta parameter, $\beta$ \\
		The beta parameter defined as $\beta := \frac{\displaystyle 8 \pi n T}{\displaystyle B^{2}}$ describes the ratio of the thermal and magnetic energies of the plasma, as particles in random thermal motions compete with the Lorentz force.
		
	\end{enumerate} 
		Many other parameters describe a plasma including several electrodynamic quantities.
	
	\subsection{Laboratory Plasma}
	Laboratory plasmas are often characterized by some properties like:
	\begin{itemize}[itemsep=0cm]
		\item High ionization fraction
		\item Sub atmospheric pressure required to sustain ionization
		\item Temperature range : 1000 - 30000 K
	\end{itemize}

	\noindent Many surface engineering processes use plasmas to obtain high performance coatings. Some processes that use plasma are:
	\begin{enumerate}[itemsep=0cm]
		\item Plasma Immersion Ion Implantation
		\item Plasma Enhanced Chemical Vapor Deposition
		\item Magnetron Sputtering (PVD)
		\item Air Plasma Spray (Spray technique)
		\item Plasma Transferred Arc (Hardfacing technique)
	\end{enumerate}
	
	\subsection{Physical Vapor Deposition}
	Physical Vapor Deposition(PVD) is a family of surface engineering techniques where thin film coatings are grown on the surface of a specimen, in a vacuum chamber. Particles from a vapor move around on the surface of the specimen as random walkers and eventually get trapped in strained pockets producing nucleation site for the growth of a film. Most PVD techniques fall under either of the two catgories:
	\begin{enumerate}
		\item Evaporation techniques \\
		Evaporation techniques involve heating the material to  to a high temperature when it forms vapor and the particles in the vapor are coated on the substrate. It is for that reason that evaporation techniques are called hot techniques. Semiconductors like Si/Ge, insulators like oxides and metals like Tungsten are often coated on substrates using evaporation techniques. 
		
		\item Sputtering techniques \\
		Sputtering techniques on the other hand are known as cold techniques. Sputtering techniques use a high energy beam to remove material from a target and the removed material is coated on the required substrate.
		
	\end{enumerate}

	\subsection{Magnetron Sputtering}
	Magnetron sputtering is a sputtering technique where the sputtered ions form a magnetically confined plasma. The plasma, controlled by magnetic fields, transports the ions to the surface of the substrate forming the coating. 
	
	
	\begin{figure}[H]
		\begin{center}
			\includegraphics[width=12cm, height=8cm]{magnetron sputtering.png} \caption{Magnetron Sputtering system. credit: \cite{SurfaceEng}}
		\end{center}
	\end{figure}
	
	\subsection{Magnetic Mirror}
	The magnetic mirror effect can first be illustrated with a 	single particle. For the simple case, it is assumed that there is no electric field acting on the particle, the magnetic field is cylindrically symmetric and the gradient of the magnetic field is only along the axial direction of the cylinder, the $z$ direction.  The magnetic moment $$\mu = \frac{m v_{\perp}^{2}}{2B_{z}} $$ of the particle is an adiabatic invariant of the particle motion and hence is approximately conserved for a magnetic field whose $z$ component $B_{z}$ does not vary too much . The kinetic energy$$ KE = \frac{1}{2}m v^{2}$$ of the particle is also approximately conserved. Writing $v_{\perp} = v sin \theta$,
	$$\frac{\mu}{KE} = \frac{sin^{2}\theta}{B_{z}}$$ is also a conserved quantity. This means that as $B$ increases (within a small range), $sin^{2}\theta$ increases as well and so does $|sin\theta|$ which means that $|v_{\perp}|$ increases and since $v$ is conserved, $v_{\parallel}$ decreases to zero. When $v_{\parallel} = 0$, the particle is no longer moving in the $z$ direction, but is moving with velocity $v$ in the plane perpendicular to $z$ direction. The $v_{\parallel}$ then increases in the opposite direction as the particle moves towards decreasing $B$ because of the Lorentz force. Such a particle is seen as being reflected due to the configuration of the magnetic field and hence such a configuration of the magnetic field is called a magnetic mirror. Particles whose $v_{\parallel}$ does not decrease to 0 while the value of $B$ is decreasing, escape from the magnetic mirror configuration. \\
	
	\noindent A magnetic mirror configuration is important in a magnetic plasma trap chamber such as that used in Magnetron sputtering. Particles in a plasma have different speeds depending on the initial distribution which is based on parameters like the plasma temperature. The speeds of particles change depending on the electric and magnetic fields. Based on the magnetic mirror effect, one can determine which particle (having certain velocities) can escape the magnetic trap and which of those are reflected. The less the particles escape the magnetic trap, the more of the flux is used in forming coatings and less of the ionized gas is wasted. This is very useful in understanding the required gas supply and rate of deposition.
	
	\section{Literature Review}
	A large part of the literature review for the project consisted of studying basic plasma physics in order to understand how we could formulate the project and proceed ahead. 
	
	\subsection{Plasma as a system}
	Various models are used to describe Plasma as a system. Some of the common approaches are:
	\begin{enumerate}
		\item \textbf{Single particle description} \\
		This model is used to describe the motion of a charged particle under the influence of electric and magnetic fields. The particle's motion is described by the Lorentz force 
		\begin{equation}
			\label{eqn:lorentz}
			\frac{d \textbf{$\boldsymbol{v}$}}{d t} = \frac{q}{m} \left(\textbf{$\mathbf{E}$} + \textbf{$\boldsymbol{v}$} \times \textbf{$\mathbf{B}$} \right)
		\end{equation}
		Describing many particles evolving under the influence of Lorentz force, it is easy to describe a plasma assuming that the mean free path of the particles is much larger than the dimensions of the chamber; meaning that particles hardly ever collide. This is to say that in the simplest case, under this model particles evolve under the influence of electric and magnetic fields but the particles do not produce any electric and magnetic fields of their own or affect the external applied fields and hence also do not interact with other particles.
	
		\item \textbf{Kinetic theory} \\
		The kinetic theory describes the plasma as collection of particles whose state (position and velocity) is treated as a random variable with a density function \\ $ f(x, y, z, v_{x}, v_{y}, v_{z}, t) $ which describes the number of particles at position $ (x, y, z) $ at time $ t $ with velocities between $ v_{x} $ and $ v_{x} + dv_{x} $, $ v_{y} $ and $ v_{y} + dv_{y} $, $ v_{z} $ and $ v_{z} + dv_{z} $ in directions $x$, $y$ and $z$ respectively. For a simpler notation $ f(x, y, z, v_{x}, v_{y}, v_{z}, t) $ is denoted as $ f(\boldsymbol{r}, \boldsymbol{v}, t) $.
		The expression $$\displaystyle \int_{all \hspace{0.2cm} v_{x}^{}} dv_{x} \int_{all \hspace{0.2cm} v_{y}^{}} dv_{y} \int_{all \hspace{0.2cm} v_{z}^{}} dv_{z} f(\boldsymbol{r}, \boldsymbol{v}, t) $$ gives the number of particles at position $\boldsymbol{r}$, at time $t$.  For a simple choice of notation, it if often written as $$\displaystyle \int_{all \hspace{0.2cm} \boldsymbol{v}^{}} d^{3}v f(\boldsymbol{r}, \boldsymbol{v}, t) \hspace{0.5cm}\mathnormal{or} \hspace{0.5cm}\int_{all \hspace{0.2cm} \boldsymbol{v}^{}} d\boldsymbol{v} f(\boldsymbol{r}, \boldsymbol{v}, t)$$ A density function is said to be normalized if $$\displaystyle \int_{all \hspace{0.2cm} \boldsymbol{v}^{}} d\boldsymbol{v} f(\boldsymbol{r}, \boldsymbol{v}, t) = 1$$ Such a density function is also denoted with a hat as $ \hat{f}(\boldsymbol{r}, \boldsymbol{v}, t) $.\\
		\noindent The average velocity $\bar{v}$ for a density function $ \hat{f}(\boldsymbol{r}, \boldsymbol{v}, t) $ is calculated as  $$\displaystyle \int_{all \hspace{0.2cm} \boldsymbol{v}^{}} d\boldsymbol{v} \hspace{0.2cm} v f(\boldsymbol{r}, \boldsymbol{v}, t)$$ Various other features of the distribution are: the Root Mean Square velocity $v_{rms}$, the average absolute velocity $|\bar{v}|$, the average velocity in $z$ direction $\bar{v_{z}}$, etc.\\
		The evolution of the particles is described as the changing of the density function. Boltzmann equation is often used to describe this dynamics:
		
		$$\frac{\partial f}{\partial t} + \boldsymbol{v} \cdot \nabla f + \frac{\mathbf{F}}{m} \cdot {\partial}_{\boldsymbol{v}} f = \left(\frac{\partial f}{\partial t}\right)_{c}$$
		\item \textbf{Fluid model} \\
		The fluid model is describes the plasma in terms of macroscopic variables like Pressure, Temperature, average velocity, density, flux; like ordinary fluid dynamics. The governing Partial Differential Equations are obtained by taking moments of the Boltzmann equation, or a special case of the Boltzmann equation called the Vlasov equation. Such a procedure gives rise to: \\
		
		The continuity equation: \\
		$$\frac{\partial \varrho}{\partial t} + \boldsymbol{\nabla} \cdot \left(\varrho \boldsymbol{v}\right) = 0$$
		and the Cauchy momentum equation: \\
		$$\varrho \left(\frac{\partial}{\partial t} + \boldsymbol{v} \cdot \boldsymbol{\nabla}  \right) \boldsymbol{v} = \mathbf{J} \times \mathbf{B} - \boldsymbol{\nabla} p$$
		Higher moments yield other equations for quantities like the entropy. The fluid equations average over the velocity distribution of the particles to obtain the macroscopic variables like Pressure and Temperature, as discussed earlier- by taking moments of the Kinetic equations like the Boltzmann equation or the Vlasov equation.
		
		\item \textbf{Magnetohydrodynamics} \\
		Magnetohydrodynamics(MHD) is an extension of the fluid model of describing a plasma. In a first approximation, the magnetic and electric fields acting on the fluid and the currents generated are now solved using Faraday's law: \\
		$$\frac{\partial \mathbf{B}}{\partial t} = - \boldsymbol{\nabla} \times \mathbf{E}$$
		and the Ampere's law: \\
		$$\mu_{0} \mathbf{J} = \boldsymbol{\nabla} \times \mathbf{B}$$
		The full set of Maxwell's equations can also be coupled with the fluid. MHD description of a plasma also describes dynamic waves such as Alfv\'en waves and magnetosonic waves. MHD theory can also be used to describe interaction of the plasma with electromagnetic waves, for example those from a LASER source relevant to LASER plasma processes in surface engineering. Ideal MHD studies plasma as a fluid of single species, however, multi-species plasma can also be studied. \\
		
		\item \textbf{The model used in the project} \\
		In the context of plasma processes like Magnetron Sputtering where particles have a considerably large mean free path, we have decided not to use the fluid model and the MHD descriptions of a plasma. In such processes, it is almost always the case that the strength of the electric and magnetic fields applied by the apparatus is far more stronger than the electric and magnetic fields generated by charged particles in a plasma. This approximation allows us to avoid solving Maxwell's equations and use only the Lorentz force. The model used in the project is mostly similar to the single particle model, where in a collection of particles, each particle evolves under the influence of the electric and magnetic fields set up by the apparatus; governed by the Lorentz force. However, we incorporate a little bit of the Kinetic theory in that we are interested in different initial velocity distributions for particles in the plasma and how the velocity distribution changes over time; as it is important to understand the velocity distribution to characterize the reflection and loss of particles in a magnetic-mirror like trap that may be setup in the plasma chamber.  
		
	\end{enumerate}
	
	\section{Gaps in Literature}
	It would take a number of courses on plasma physics to fully understand the current research methods in plasma physics. In the interest of time, we have studied some simple setups; so that we can setup well functioning plasma configuration during the course of the project and study some aspects that we are interested in. The forefront of research in plasma physics concerns complicated devices like Magnetic Confinement Fusion, or Quantum Optic systems. Most research in surface engineering focuses on the properties of coatings obtained and parameters of plasma used; in processes that use plasma. In our project, we would like to set up a simple plasma simulation that can help us study smaller devices like the one available in the School of Mechanical Engineering; where we can control a small flux of particles by tuning the electric and magnetic fields. This section describes, how we construct a simple easy to use, plasma simulation system; which is yet to be functional.
	
	\subsection{Particle in Cell Methods}
	Particle in a cell methods are used to simulate the kinetic theory of plasma. A simple strategy used for particle in cell plasma simulation based on strategies as outlined in the paper \cite{PIC good} and the slides \cite{PIC IAS} involves the following steps:
	\begin{enumerate}
		\item \textbf{Sampling and Initialization} \\
		The initial positions and velocities of particles in the plasma are sampled from a distribution, or based on some strategy.
		\item \textbf{Action of fields on the particles} \\
		The particles move under the influence of electric and magnetic fields as described by the Lorentz force as stated in the equation \eqref{eqn:lorentz}.
		\item \textbf{Particle deposition} \\
		In this step, charged particles are deposited on the grid defined by the mesh, and the charge density $\varrho_{i}$ and the current density $\boldsymbol{j}_{i}$ generated by the deposited particles is computed. One strategy outlined in the paper \cite{PIC good} defines charge deposition as following.
		$\boldsymbol{x_{i}} = \left(\boldsymbol{i} + 1/2\right) \Delta \boldsymbol{x}$, $\boldsymbol{i} \in \mathbb{Z}^{D}$ define the grid. A second order deposition can be achieved by:
		$$\varrho_{i} = \sum_{p}^{}\left(\frac{q_{p}}{V_{i}}\right) \mathbf{W_{2}} \left(\frac{\boldsymbol{x_{i}} - \boldsymbol{x_{p}}}{\Delta x}\right)$$
		where $V_{i} = \Delta x^{D}$ is the volume of the cell $i$ and $\mathbf{W_{2}}\left(\boldsymbol{x}\right)$ is a $D$-dimensional interpolating function defined in \cite{PIC good}. In simple models, the current density $\boldsymbol{j}_{i}$ is often not used.
		\item \textbf{Fields generated by particles} \\
		In this step, the electric and magnetic fields generated by the charge density and current density are computed. The paper \cite{PIC good} uses Poisson equation to compute the electric field generated by the charge distribution and neglects the magnetic field generated. However, in high performance simulations like that outlined in \cite{PIC IAS}, the full set of Maxwell's equations are used to compute the electric and magnetic fields generated by the particles.
		\item \textbf{Force on particles} \\
		In this step, the force on the particles due to the electric and magnetic fields are computed. Most simulations like the one outlined in the paper \cite{PIC good}; because they compute the electric and magnetic fields generated by the deposited particles, are able to describe the interaction of each particle with the electric and magnetic fields generated by other particles in the plasma, and hence capture the particle-particle dynamics.
		\item \textbf{Action of the force on particles} \\
		Step 2 is repeated to move the particles under the influence of the electric and magnetic fields.
	\end{enumerate}

	A flow-chart for the simulation based on a similar strategy outlined in \cite{PIC IAS} is presented below.

	\begin{figure}[H]
		\begin{center}
			\includegraphics[width=12cm, height=8cm]{PIC.png} \caption{Flowchart for PIC methods. credit: \cite{PIC IAS}}
		\end{center}
	\end{figure}
	
	\section{Problem Definition}
	Our approach is similar to and differs from the general procedure outlined for Particle in Cell methods, in the following ways:
	\begin{enumerate}
		\item \textbf{Sampling and Initialization} \hspace*{\fill}Similar \phantom{text} \\
		This step is similar to the first step in general PIC methods. We sample the initial positions and velocities of the particles based on some strategies like sampling from the Maxwell-Boltzmann distribution or uniform initialization strategies described in Methodology.
		\item \textbf{Action of fields on the particles} \hspace*{\fill}Similar \phantom{text}\\
		This step is similar to the second step in general PIC methods. We update the positions and particles of the particles according to the Boris Algorithm, based on the Lorentz force; as described in Methodology.
		\item \textbf{Particle deposition} \hspace*{\fill}Skipped \phantom{text}\\
		We skip this step in our approach for reasons described in the next step.
		\item \textbf{Fields generated} \hspace*{\fill}Different \phantom{text} \\
		As discussed earlier; when discussing the model used in the project, in devices such as those used in surface engineering processes, it is often the case that the electric and magnetic fields generated by the apparatus are far more stronger than those generated by the particles. So it is reasonable to assume that the fields generated by the particles are negligible compared to the fields generated by the apparatus. While general PIC methods compute fields generated by particles in the plasma, we only used fields generated by the apparatus. Since we neglect the fields generated by the particles in the plasma, we can skip depositing particles in the grid; which would be the third step of general outline of PIC methods. This makes our model simpler and easier to work with, evaluate and understand. We define the fields generated by the apparatus or field configurations based on analytic configurations and move the particles in the plasma under their influence. This is later described in methodology. 
		\item \textbf{Force on particles} \hspace*{\fill}Skipped \phantom{text} \\
		As we do not compute the electric and magnetic fields generated by the particles, this step is skipped and the action of the electric and magnetic fields created by the apparatus is done directly in step 6. 
		\item \textbf{Action of the fields on particles} \hspace*{\fill}Similar \phantom{text} \\
		Similar to the sixth step of general PIC methods, we repeat step 2 to move the particles under the influence of electric and magnetic fields created by the apparatus.
	\end{enumerate}

	\noindent Our approach can be summarized in the following set of steps: \\
	\phantom{te} Do for each batch of particles in the plasma stream:
	\begin{enumerate}[itemsep=0cm]
		\item \textbf{Sampling and Initialization} \hspace*{\fill}Kinetic Theory \phantom{text}
		\item \textbf{Definition of fields} \hspace*{\fill}Apparatus \phantom{text} \\
		If required use a different field to simulate control of the apparatus, for example: changing the voltage of the electrode; changing the electric field.
		\vspace{0.2cm} \\Do for certain number of time steps:
		
		\begin{itemize}[itemsep=0cm]
			\item \textbf{Particle update based on Lorentz force} \hspace*{\fill}Single particle dynamics \phantom{text}
		\end{itemize}
		\item \textbf{Remove particles} \hspace*{\fill}Surface Engineering process \phantom{text}\\
		Particles are either absorbed to form a coating or exit the plasma chamber.
	\end{enumerate}
	The details are described later in methodology. The steps of our approach as described help us define our objectives as discussed in the following section.
	
	\section{Objectives}
	The objectives of the project are based on our approach to the problem as described in problem definition:
	\begin{enumerate}
		\item \textbf{Single Particle Method} \\ 
		To simulate charged particles that evolve under the influence of electric and magnetic fields; as governed by the Lorentz force.
		\item \textbf{Field Configurations} \\
		To simulate a few different configurations of electric and magnetic fields- some describing apparatus like coils; some describing analytic expressions for fields and to study the different evolution of particles.
		\item \textbf{Kinetic Theory} \\
		To study different initial velocity distributions and how the velocity distribution of particles changes as the particles evolve. Parameters like the Plasma temperature are to be studied under this topic.
		\item \textbf{Analysis} \\
		To analyze different batches or collections of particles, subjected to different field configurations.
		
	\end{enumerate}
	
	\section{Methodology}
	
	
	\subsection{Particle Sampling}
	Maxwell Boltzmann distribution and Parabolic density functions from forreport.tex file about Kinetic theory. \textbf{Copy the references and find some others}
	
	
	\subsection{Particle Evolution}
	Lorentz force and Boris Algorithm from lorentz.tex file. \textbf{Copy the references and find some others}
	
	\subsection{Fields}
	
	\subsection{Running different batches}
	
	\section{Work carried out so far}
	
	\textbf{Algorithm Outline} \\
	Mostly from algorithm.tex file. \textbf{Copy the references and find some others. May include some github repositories as references as well.}
	
	\section{Work to be done}
	
	\section{Gant Chart(Work Plan)}
	
	\section{Milestones in the project phase}
	
	\begin{thebibliography}{}
		\bibitem{SurfaceEng}
		Professor Sitaram Dash. (Fall Semester 2021). \textit{MEE4005 Surface Engineering} (lecture notes). SMEC, VIT Vellore.
		\bibitem{MKunz}
		Matthew W. Kunz. (November 9, 2020). \textit{Introduction to Plasma Astrophysics} (lecture notes). Princeton Plasma Physics Laboratory.
		\bibitem{chenbook}
		Chen, F. F. (1984). \textit{Introduction to plasma physics and controlled fusion} (Vol. 1, pp. 8-11). New York: Plenum press.
		\bibitem{mirror1}
		Na, Yong-Su (2017). \textit{Introduction to nuclear fusion} (Lecture 9 Mirror, lecture slide). Seoul National University Open Courseware.
		\bibitem{mirror2}
		F\"{o}rel\"{a}sning (2009). \textit{Charged particle motion in magnetic field} (lecture slide). Lule\"{a} University of Technology.
		\bibitem{PIC good}
		Myers, A., Colella, P., \& Straalen, B. V. (2017). \textit{A 4th-Order Particle-in-Cell Method with Phase-Space Remapping for the Vlasov--Poisson Equation}. SIAM Journal on Scientific Computing, 39(3), B467-B485.
		\bibitem{PIC IAS}
		Anatoly Spitkovsky (2016). \textit{Kinetic plasma simulations} (lecture slide). PiTP 2016 on Computational Plasma Astrophysics. Institute for Advanced Study.
		\bibitem{Borisgood}
		Qin, H., Zhang, S., Xiao, J., $\&$ Tang, W. M. (April, 2013). \textit{Why is Boris algorithm so good?}. Princeton Plasma Physics Laboratory, PPPL-4872.
		
	\end{thebibliography}

\end{document}