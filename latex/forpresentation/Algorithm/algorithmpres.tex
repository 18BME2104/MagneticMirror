\documentclass{beamer}

\usepackage{multicol}
\usepackage{float}
\usepackage{lmodern}

\usetheme[progressbar=frametitle]{metropolis}
\setbeamertemplate{frame numbering}[fraction]
\useinnertheme{metropolis}
\useoutertheme{metropolis}
\usefonttheme{metropolis}
\usecolortheme{spruce}
\setbeamercolor{background canvas}{bg=white}
%different themes are available
\definecolor{babyblueeyes}{rgb}{0.63, 0.79, 0.95}
%define custom theme using rgb, check latexcolor.com for information
\usecolortheme[named=babyblueeyes]{structure}


% Python style for highlighting
\usepackage[utf8]{inputenc}

% Default fixed font does not support bold face
\DeclareFixedFont{\ttb}{T1}{txtt}{bx}{n}{12} % for bold
\DeclareFixedFont{\ttm}{T1}{txtt}{m}{n}{12}  % for normal

% Custom colors
\usepackage{color}
\definecolor{deepblue}{rgb}{0,0,0.5}
\definecolor{deepred}{rgb}{0.6,0,0}
\definecolor{deepgreen}{rgb}{0,0.5,0}
\definecolor{almostwhite}{rgb}{0.99,0.99, 0.99}

\usepackage{listings}

% Python style for highlighting
% Check out how to use minted next time
% Seems simpler

\lstset{
	basicstyle  =   \footnotesize,
	keywordstyle    = \color{deepred}\bfseries,
	stringstyle     = \color{strings},
	identifierstyle = \color{black},
	commentstyle    = \color{deepgreen},
	breaklines=true,
	numbersep=-10pt,
	stepnumber=1,
	showspaces=false,
	escapechar=§,
	showstringspaces=false,
	showtabs=false,
	frame=single,  
	rulecolor=\color{black},
	tabsize=4,              
	captionpos=t,           
	breaklines=true,
	breakatwhitespace=false,
	numbers=left,
	extendedchars=\true,
	emph=[3]{href, Particle, Boris_update, Field, uniform_E_field, radial_E_field, uniform_B_field, helmholtz_coil_B_field, two_helmholtz_B_field, Sampler, sample_uniform_position, sample_uniform_velocity, sample_velocity_uniformKE, sample_Maxwellian_velocity, sample_parabolic_velocity},
	emphstyle=[3]{\color{deepblue}},
	backgroundcolor=\color{almostwhite},
	language=Python 
}

%until here


\title[]{Magnetic Mirror Effect in Magnetron Plasma:}
\subtitle{Modeling of Plasma Parameters}


%may use macros to set new frame defaults

\setbeamercovered{transparent=25}
%hide onslide part by making it more transparent
%cant do this with only, but with onslide


\newcommand\myfigure[1]{%
	\medskip\noindent\begin{minipage}{\columnwidth}
		\centering%
		#1%
		%figure,caption, and label go here
	\end{minipage}\medskip}
\newcommand{\comment}[1]{}

\begin{document}
	\metroset{block=fill}
	%shades the background of a block
	
	%In beamer we create frames 1 frame= 1 page to hold our information
	\begin{frame}
		\titlepage	
	\end{frame}	
	
	\begin{frame}
		\begin{block}{GitHub repository:} 
		 \phantom{a} \\
		 \hspace{1.5cm}
		 \textbf{\textit{https://github.com/18BME2104/MagneticMirror}} \\
		\phantom{a}
		\end{block}
	\vspace{0.5cm}
		\begin{block}{Required functionalities and Files}
			\begin{enumerate}
				\item Constants - \textbf{constants.ipynb}
				\item Particle - \textbf{particle.ipynb}
				\item Electric and Magnetic fields - \textbf{field.ipynb}
				\item Particle initialization - \textbf{sampling.ipynb}
				\item Updating the particles - \textbf{step.ipynb}
				\item Batches of updates - \textbf{run.ipynb}
				\item Plotting - \textbf{plot.ipynb}
			\end{enumerate}	
		\end{block}
		
	\end{frame}

	\begin{frame}[t]{1. Constants - constants.ipynb}
		Define constants
		\lstinputlisting[language=Python, linerange={1-6}]{constants.py}
		Useful constants: \\
		\hspace{0.5cm} electron charge, electron mass, ion masses, ion charges \\ \hspace{0.5cm} atomic mass unit, Avogadro's number, \\ \hspace{0.5cm} permittivity of vacuum,  permeability of vacuum, \\ \hspace{0.5cm} Boltzmann's constant
	\end{frame}
	\begin{frame}[t]{2. Particle - particle.ipynb}
		Define state of a particle and update it \\
		\vspace{1cm}
		\begin{enumerate}
			\item Particle State: \\
			Position and Velocity \\
			name, mass, charge
			\item Update State: \\
			Boris Update \\
			... other updates strategies
		\end{enumerate}
	\end{frame}
	
	\begin{frame}
		\lstinputlisting[language=Python, linerange={1-13}]{particle.py}
	\end{frame}
	\begin{frame}
		\lstinputlisting[language=Python, linerange={14-28}]{particle.py}
	\end{frame}

	\begin{frame}[t]{3. Electric and Magnetic fields - field.ipynb}
		Define field configurations \\
		\vspace{0.5cm}
		\begin{itemize}
			\item Electric Field:
			\begin{enumerate}
				\item Uniform Electric Field
				\item Radial Electric Field
			\end{enumerate}
			\item Magnetic Field: 
			\begin{enumerate}
				\item Uniform Magnetic Field
				\item Magnetic Field due to Helmholtz Coil
				\item Magnetic Field due to 2 Helmholtz Coils
			\end{enumerate}
		\end{itemize}
	... other field configurations:
	\begin{enumerate}
		\item Analytic field expression
		\item Apparatus like coils
	\end{enumerate}
	\end{frame}
	\begin{frame}
		\lstinputlisting[language=Python, linerange={1-13}]{field.py}
	\end{frame}
	\begin{frame}
		\lstinputlisting[language=Python, linerange={1-1, 14-22, 25-27}]{field.py}
	\end{frame}

	\begin{frame}[t]{4. Particle initialization - sampling.ipynb}
		Sample initial positions and velocities \\
		\begin{itemize}
			\item Initial Position Samping:
			\begin{enumerate}
				\item Same position for all: eg: valve position
				\item Same distance from a point: eg: from an electrode \\
				... other distributions
			\end{enumerate}
			\item Initial Velocity Sampling:
			\begin{enumerate}
				\item Same velocity for all: eg: injected by pump
				\item Same speed for all
				\item Speeds based on same K.E. : \\ eg: accelerated through same potential
				\item Maxwellian distributed speeds
				\item Maxwellian distributed velocities
				\item Parabolic distribution \\
				... other distributions
				
			\end{enumerate}
			
			\item Writing to files \\
			\hspace{0.5cm}csv format 
		\end{itemize}
		
	\end{frame}

	\begin{frame}
		\lstinputlisting[language=Python, linerange={1-9, 26-30}]{sampling.py}
	\end{frame}
	\begin{frame}
		\lstinputlisting[language=Python, linerange={1-2, 56-65}]{sampling.py}
	\end{frame}
	\begin{frame}
		\lstinputlisting[language=Python, linerange={1-3, 67-77}]{sampling.py}
	\end{frame}

	\begin{frame}[t]{5. Updating the particles - step.ipynb}
		Evolve particles under influence of fields
		\vspace{0.5cm}
		\begin{enumerate}
			\item Initialize particles: hold states \\
			\begin{itemize}
				\item Use initial positions and velocities directly
				\item Read initial positions and velocities from saved files
			\end{itemize}
			\item Initialize fields: call field configurations
			\item Update particles: call updates
		\end{enumerate}
	\end{frame}
	\begin{frame}
		\lstinputlisting[language=Python, linerange={1-13}]{step.py}
	\end{frame}
	\begin{frame}
		\lstinputlisting[language=Python, linerange={15-20, 41-41}]{step.py}
	\end{frame}
	\begin{frame}
		\lstinputlisting[language=Python, linerange={1-3, 21-33 }]{step.py}
	\end{frame}

	\begin{frame}[t]{6. Batches of updates - run.ipynb}
		Define dynamic plasma system
		\vspace{0.5cm}
		\begin{enumerate}
			\item Create particles
			\item Update particles
			\item Update fields
			\item Create new particles
			\item Remove particles
		\end{enumerate}
	\end{frame}
	\begin{frame}
		\lstinputlisting[language=Python, linerange={1-1, 4-13}]{run.py}
	\end{frame}
	\begin{frame}
		\lstinputlisting[language=Python, linerange={15-21}]{run.py}
	\end{frame}

	\begin{frame}[t]{7. Plotting - plot.ipynb}
		Plot functionalities
	\end{frame}

	\begin{frame}[t]{References}
	\begin{thebibliography}{}
		\bibitem{Borisgood}
		Qin, H., Zhang, S., Xiao, J., $\&$ Tang, W. M. (April, 2013). \textit{Why is Boris algorithm so good?}. Princeton Plasma Physics Laboratory, PPPL-4872.
		
	\end{thebibliography}
	\end{frame}

\end{document}