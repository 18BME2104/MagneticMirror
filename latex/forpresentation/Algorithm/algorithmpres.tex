\documentclass{beamer}

\usepackage{multicol}
\usepackage{float}

\usetheme[progressbar=frametitle]{metropolis}
\setbeamertemplate{frame numbering}[fraction]
\useinnertheme{metropolis}
\useoutertheme{metropolis}
\usefonttheme{metropolis}
\usecolortheme{spruce}
\setbeamercolor{background canvas}{bg=white}
%different themes are available
\definecolor{babyblueeyes}{rgb}{0.63, 0.79, 0.95}
%define custom theme using rgb, check latexcolor.com for information
\usecolortheme[named=babyblueeyes]{structure}




\title[]{Magnetic Mirror Effect in Magnetron Plasma:}
\subtitle{Modeling of Plasma Parameters}


%may use macros to set new frame defaults

\setbeamercovered{transparent=25}
%hide onslide part by making it more transparent
%cant do this with only, but with onslide


\newcommand\myfigure[1]{%
	\medskip\noindent\begin{minipage}{\columnwidth}
		\centering%
		#1%
		%figure,caption, and label go here
	\end{minipage}\medskip}
\newcommand{\comment}[1]{}

\begin{document}
	\metroset{block=fill}
	%shades the background of a block
	
	%In beamer we create frames 1 frame= 1 page to hold our information
	\begin{frame}
		\titlepage	
	\end{frame}	
	
	\begin{frame}[t]{1. Control on Particle Update strategies - NOT a control knob }
		Not really a control knob. Concerns precision of solution / update. \\
		
		\noindent Update strategies used in \textbf{particle.ipynb}\\
		
		\noindent Describe the Boris algorithm update strategy.
	\end{frame}

	\begin{frame}[t]{2. Control of Electric and Magnetic fields - Control Knobs here}
		Electric and Magnetic field configurations described in \textbf{field.ipynb} \\
		
		\noindent Describe the Helmholtz coil magnetic field and electrode potential electric field configurations used. \\
		
		\noindent
		Different Electric field configurations could be used. Simple example: changing the electrode voltages. \\
		
		\noindent Different Magnetic field configurations could be used. Simple example: using many Helmholtz coils (number controllable), at different angles (angle controllable).
	\end{frame}

	\begin{frame}[t]{3. Controlling particle initialization - Control knobs here}
		Sampling particles with different initial velocities, and positions for example using different density functions f. For example: based on parameters like plasma Temperature. \\
		
		\noindent Different particle sampling and initialization strategies used in \textbf{interaction.ipynb}
		
		Also track how the velocity distribution changes with time.
	\end{frame}

	\begin{frame}[t]{References}
	\begin{thebibliography}{}
		\bibitem{Borisgood}
		Qin, H., Zhang, S., Xiao, J., $\&$ Tang, W. M. (April, 2013). \textit{Why is Boris algorithm so good?}. Princeton Plasma Physics Laboratory, PPPL-4872.
		
	\end{thebibliography}
	\end{frame}

\end{document}