\documentclass{beamer}

\usepackage{multicol}
\usepackage{float}

\usetheme[progressbar=frametitle]{metropolis}
\setbeamertemplate{frame numbering}[fraction]
\useinnertheme{metropolis}
\useoutertheme{metropolis}
\usefonttheme{metropolis}
\usecolortheme{spruce}
\setbeamercolor{background canvas}{bg=white}
%different themes are available
\definecolor{babyblueeyes}{rgb}{0.63, 0.79, 0.95}
%define custom theme using rgb, check latexcolor.com for information
\usecolortheme[named=babyblueeyes]{structure}




\title[Interacting Manifolds]{Magnetic Mirror Effect in Magnetron Plasma:}
\subtitle{Modeling of Plasma Parameters}


%may use macros to set new frame defaults

\setbeamercovered{transparent=25}
%hide onslide part by making it more transparent
%cant do this with only, but with onslide


\newcommand\myfigure[1]{%
	\medskip\noindent\begin{minipage}{\columnwidth}
		\centering%
		#1%
		%figure,caption, and label go here
	\end{minipage}\medskip}
\newcommand{\comment}[1]{}

\begin{document}
	\metroset{block=fill}
	%shades the background of a block
	
	%In beamer we create frames 1 frame= 1 page to hold our information
	\begin{frame}
		\titlepage	
	\end{frame}	
	
	\begin{frame}[t]{1. Lorentz Force }
	
	\begin{block}{Lorentz force}
		\begin{equation}
			\label{eqn:lorentz}
			\frac{d \textbf{v}}{d t} = \frac{q}{m} \left(\textbf{E} + \textbf{v} \times \textbf{B} \right)
		\end{equation}
	
	\end{block}
		\begin{equation}
			\label{eqn:velocity}
			\frac{d \textbf{x}}{d t} = \textbf{v}
		\end{equation}
	
	\begin{block}{Discretized Lorentz force}
		\begin{equation}
			\label{eqn:Dlorentz}
			\frac{\textbf{v}_{k+1} - \textbf{v}_{k}}{\Delta t} = \frac{q}{m} \left[\textbf{E}_{k} + \frac{\left( \textbf{v}_{k+1} + \textbf{v}_{k} \right) }{2} \times \textbf{B}_{k} \right] 
		\end{equation}
	\end{block}

	\begin{equation}
		\label{eqn:Dvelocity}
		\frac{\textbf{x}_{k+1} - \textbf{x}_{k} }{\Delta t} =	\textbf{v}_{k+1}
	\end{equation}

	\end{frame}
	
	\begin{frame}[t]{2. Boris Algorithm}
		\begin{block}{Boris Algorithm}
			\begin{equation}
				\begin{split}
					\textbf{v}^{-} & = \textbf{v}_{k} + q^{\prime} \textbf{E}_{k} \\
					\textbf{v}^{+} & = \textbf{v}^{-} + 2 q^{\prime} \left( \textbf{v}^{-} \times \textbf{B}_{k} \right) \\
					\textbf{v}_{k+1} & = \textbf{v}^{+} + q^{\prime} \textbf{E}_{k} \\
					\textbf{x}_{k+1} & = \textbf{x}_{k} + \Delta t \hspace{0.2cm}\textbf{v}_{k+1}
				\end{split}
			\end{equation}
		\end{block}
		
		where $q^{\prime} = \frac{\displaystyle q}{\displaystyle m} \frac{\displaystyle \Delta t}{ 2}$.
	\end{frame}

	\begin{frame}[t]{References}
	\begin{thebibliography}{}
		\bibitem{Borisgood}
		Qin, H., Zhang, S., Xiao, J., $\&$ Tang, W. M. (April, 2013). \textit{Why is Boris algorithm so good?}. Princeton Plasma Physics Laboratory, PPPL-4872.
		
	\end{thebibliography}
	\end{frame}

\end{document}