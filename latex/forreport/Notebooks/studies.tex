\documentclass[11pt]{article}

    \usepackage[breakable]{tcolorbox}
    \usepackage{parskip} % Stop auto-indenting (to mimic markdown behaviour)
    
    \usepackage{iftex}
    \ifPDFTeX
    	\usepackage[T1]{fontenc}
    	\usepackage{mathpazo}
    \else
    	\usepackage{fontspec}
    \fi

    % Basic figure setup, for now with no caption control since it's done
    % automatically by Pandoc (which extracts ![](path) syntax from Markdown).
    \usepackage{graphicx}
    % Maintain compatibility with old templates. Remove in nbconvert 6.0
    \let\Oldincludegraphics\includegraphics
    % Ensure that by default, figures have no caption (until we provide a
    % proper Figure object with a Caption API and a way to capture that
    % in the conversion process - todo).
    \usepackage{caption}
    \DeclareCaptionFormat{nocaption}{}
    \captionsetup{format=nocaption,aboveskip=0pt,belowskip=0pt}

    \usepackage{float}
    \floatplacement{figure}{H} % forces figures to be placed at the correct location
    \usepackage{xcolor} % Allow colors to be defined
    \usepackage{enumerate} % Needed for markdown enumerations to work
    \usepackage{geometry} % Used to adjust the document margins
    \usepackage{amsmath} % Equations
    \usepackage{amssymb} % Equations
    \usepackage{textcomp} % defines textquotesingle
    % Hack from http://tex.stackexchange.com/a/47451/13684:
    \AtBeginDocument{%
        \def\PYZsq{\textquotesingle}% Upright quotes in Pygmentized code
    }
    \usepackage{upquote} % Upright quotes for verbatim code
    \usepackage{eurosym} % defines \euro
    \usepackage[mathletters]{ucs} % Extended unicode (utf-8) support
    \usepackage{fancyvrb} % verbatim replacement that allows latex
    \usepackage{grffile} % extends the file name processing of package graphics 
                         % to support a larger range
    \makeatletter % fix for old versions of grffile with XeLaTeX
    \@ifpackagelater{grffile}{2019/11/01}
    {
      % Do nothing on new versions
    }
    {
      \def\Gread@@xetex#1{%
        \IfFileExists{"\Gin@base".bb}%
        {\Gread@eps{\Gin@base.bb}}%
        {\Gread@@xetex@aux#1}%
      }
    }
    \makeatother
    \usepackage[Export]{adjustbox} % Used to constrain images to a maximum size
    \adjustboxset{max size={0.9\linewidth}{0.9\paperheight}}

    % The hyperref package gives us a pdf with properly built
    % internal navigation ('pdf bookmarks' for the table of contents,
    % internal cross-reference links, web links for URLs, etc.)
    \usepackage{hyperref}
    % The default LaTeX title has an obnoxious amount of whitespace. By default,
    % titling removes some of it. It also provides customization options.
    \usepackage{titling}
    \usepackage{longtable} % longtable support required by pandoc >1.10
    \usepackage{booktabs}  % table support for pandoc > 1.12.2
    \usepackage[inline]{enumitem} % IRkernel/repr support (it uses the enumerate* environment)
    \usepackage[normalem]{ulem} % ulem is needed to support strikethroughs (\sout)
                                % normalem makes italics be italics, not underlines
    \usepackage{mathrsfs}
    

    
    % Colors for the hyperref package
    \definecolor{urlcolor}{rgb}{0,.145,.698}
    \definecolor{linkcolor}{rgb}{.71,0.21,0.01}
    \definecolor{citecolor}{rgb}{.12,.54,.11}

    % ANSI colors
    \definecolor{ansi-black}{HTML}{3E424D}
    \definecolor{ansi-black-intense}{HTML}{282C36}
    \definecolor{ansi-red}{HTML}{E75C58}
    \definecolor{ansi-red-intense}{HTML}{B22B31}
    \definecolor{ansi-green}{HTML}{00A250}
    \definecolor{ansi-green-intense}{HTML}{007427}
    \definecolor{ansi-yellow}{HTML}{DDB62B}
    \definecolor{ansi-yellow-intense}{HTML}{B27D12}
    \definecolor{ansi-blue}{HTML}{208FFB}
    \definecolor{ansi-blue-intense}{HTML}{0065CA}
    \definecolor{ansi-magenta}{HTML}{D160C4}
    \definecolor{ansi-magenta-intense}{HTML}{A03196}
    \definecolor{ansi-cyan}{HTML}{60C6C8}
    \definecolor{ansi-cyan-intense}{HTML}{258F8F}
    \definecolor{ansi-white}{HTML}{C5C1B4}
    \definecolor{ansi-white-intense}{HTML}{A1A6B2}
    \definecolor{ansi-default-inverse-fg}{HTML}{FFFFFF}
    \definecolor{ansi-default-inverse-bg}{HTML}{000000}

    % common color for the border for error outputs.
    \definecolor{outerrorbackground}{HTML}{FFDFDF}

    % commands and environments needed by pandoc snippets
    % extracted from the output of `pandoc -s`
    \providecommand{\tightlist}{%
      \setlength{\itemsep}{0pt}\setlength{\parskip}{0pt}}
    \DefineVerbatimEnvironment{Highlighting}{Verbatim}{commandchars=\\\{\}}
    % Add ',fontsize=\small' for more characters per line
    \newenvironment{Shaded}{}{}
    \newcommand{\KeywordTok}[1]{\textcolor[rgb]{0.00,0.44,0.13}{\textbf{{#1}}}}
    \newcommand{\DataTypeTok}[1]{\textcolor[rgb]{0.56,0.13,0.00}{{#1}}}
    \newcommand{\DecValTok}[1]{\textcolor[rgb]{0.25,0.63,0.44}{{#1}}}
    \newcommand{\BaseNTok}[1]{\textcolor[rgb]{0.25,0.63,0.44}{{#1}}}
    \newcommand{\FloatTok}[1]{\textcolor[rgb]{0.25,0.63,0.44}{{#1}}}
    \newcommand{\CharTok}[1]{\textcolor[rgb]{0.25,0.44,0.63}{{#1}}}
    \newcommand{\StringTok}[1]{\textcolor[rgb]{0.25,0.44,0.63}{{#1}}}
    \newcommand{\CommentTok}[1]{\textcolor[rgb]{0.38,0.63,0.69}{\textit{{#1}}}}
    \newcommand{\OtherTok}[1]{\textcolor[rgb]{0.00,0.44,0.13}{{#1}}}
    \newcommand{\AlertTok}[1]{\textcolor[rgb]{1.00,0.00,0.00}{\textbf{{#1}}}}
    \newcommand{\FunctionTok}[1]{\textcolor[rgb]{0.02,0.16,0.49}{{#1}}}
    \newcommand{\RegionMarkerTok}[1]{{#1}}
    \newcommand{\ErrorTok}[1]{\textcolor[rgb]{1.00,0.00,0.00}{\textbf{{#1}}}}
    \newcommand{\NormalTok}[1]{{#1}}
    
    % Additional commands for more recent versions of Pandoc
    \newcommand{\ConstantTok}[1]{\textcolor[rgb]{0.53,0.00,0.00}{{#1}}}
    \newcommand{\SpecialCharTok}[1]{\textcolor[rgb]{0.25,0.44,0.63}{{#1}}}
    \newcommand{\VerbatimStringTok}[1]{\textcolor[rgb]{0.25,0.44,0.63}{{#1}}}
    \newcommand{\SpecialStringTok}[1]{\textcolor[rgb]{0.73,0.40,0.53}{{#1}}}
    \newcommand{\ImportTok}[1]{{#1}}
    \newcommand{\DocumentationTok}[1]{\textcolor[rgb]{0.73,0.13,0.13}{\textit{{#1}}}}
    \newcommand{\AnnotationTok}[1]{\textcolor[rgb]{0.38,0.63,0.69}{\textbf{\textit{{#1}}}}}
    \newcommand{\CommentVarTok}[1]{\textcolor[rgb]{0.38,0.63,0.69}{\textbf{\textit{{#1}}}}}
    \newcommand{\VariableTok}[1]{\textcolor[rgb]{0.10,0.09,0.49}{{#1}}}
    \newcommand{\ControlFlowTok}[1]{\textcolor[rgb]{0.00,0.44,0.13}{\textbf{{#1}}}}
    \newcommand{\OperatorTok}[1]{\textcolor[rgb]{0.40,0.40,0.40}{{#1}}}
    \newcommand{\BuiltInTok}[1]{{#1}}
    \newcommand{\ExtensionTok}[1]{{#1}}
    \newcommand{\PreprocessorTok}[1]{\textcolor[rgb]{0.74,0.48,0.00}{{#1}}}
    \newcommand{\AttributeTok}[1]{\textcolor[rgb]{0.49,0.56,0.16}{{#1}}}
    \newcommand{\InformationTok}[1]{\textcolor[rgb]{0.38,0.63,0.69}{\textbf{\textit{{#1}}}}}
    \newcommand{\WarningTok}[1]{\textcolor[rgb]{0.38,0.63,0.69}{\textbf{\textit{{#1}}}}}
    
    
    % Define a nice break command that doesn't care if a line doesn't already
    % exist.
    \def\br{\hspace*{\fill} \\* }
    % Math Jax compatibility definitions
    \def\gt{>}
    \def\lt{<}
    \let\Oldtex\TeX
    \let\Oldlatex\LaTeX
    \renewcommand{\TeX}{\textrm{\Oldtex}}
    \renewcommand{\LaTeX}{\textrm{\Oldlatex}}
    % Document parameters
    % Document title
    \title{studies}
    
    
    
    
    
% Pygments definitions
\makeatletter
\def\PY@reset{\let\PY@it=\relax \let\PY@bf=\relax%
    \let\PY@ul=\relax \let\PY@tc=\relax%
    \let\PY@bc=\relax \let\PY@ff=\relax}
\def\PY@tok#1{\csname PY@tok@#1\endcsname}
\def\PY@toks#1+{\ifx\relax#1\empty\else%
    \PY@tok{#1}\expandafter\PY@toks\fi}
\def\PY@do#1{\PY@bc{\PY@tc{\PY@ul{%
    \PY@it{\PY@bf{\PY@ff{#1}}}}}}}
\def\PY#1#2{\PY@reset\PY@toks#1+\relax+\PY@do{#2}}

\@namedef{PY@tok@w}{\def\PY@tc##1{\textcolor[rgb]{0.73,0.73,0.73}{##1}}}
\@namedef{PY@tok@c}{\let\PY@it=\textit\def\PY@tc##1{\textcolor[rgb]{0.24,0.48,0.48}{##1}}}
\@namedef{PY@tok@cp}{\def\PY@tc##1{\textcolor[rgb]{0.61,0.40,0.00}{##1}}}
\@namedef{PY@tok@k}{\let\PY@bf=\textbf\def\PY@tc##1{\textcolor[rgb]{0.00,0.50,0.00}{##1}}}
\@namedef{PY@tok@kp}{\def\PY@tc##1{\textcolor[rgb]{0.00,0.50,0.00}{##1}}}
\@namedef{PY@tok@kt}{\def\PY@tc##1{\textcolor[rgb]{0.69,0.00,0.25}{##1}}}
\@namedef{PY@tok@o}{\def\PY@tc##1{\textcolor[rgb]{0.40,0.40,0.40}{##1}}}
\@namedef{PY@tok@ow}{\let\PY@bf=\textbf\def\PY@tc##1{\textcolor[rgb]{0.67,0.13,1.00}{##1}}}
\@namedef{PY@tok@nb}{\def\PY@tc##1{\textcolor[rgb]{0.00,0.50,0.00}{##1}}}
\@namedef{PY@tok@nf}{\def\PY@tc##1{\textcolor[rgb]{0.00,0.00,1.00}{##1}}}
\@namedef{PY@tok@nc}{\let\PY@bf=\textbf\def\PY@tc##1{\textcolor[rgb]{0.00,0.00,1.00}{##1}}}
\@namedef{PY@tok@nn}{\let\PY@bf=\textbf\def\PY@tc##1{\textcolor[rgb]{0.00,0.00,1.00}{##1}}}
\@namedef{PY@tok@ne}{\let\PY@bf=\textbf\def\PY@tc##1{\textcolor[rgb]{0.80,0.25,0.22}{##1}}}
\@namedef{PY@tok@nv}{\def\PY@tc##1{\textcolor[rgb]{0.10,0.09,0.49}{##1}}}
\@namedef{PY@tok@no}{\def\PY@tc##1{\textcolor[rgb]{0.53,0.00,0.00}{##1}}}
\@namedef{PY@tok@nl}{\def\PY@tc##1{\textcolor[rgb]{0.46,0.46,0.00}{##1}}}
\@namedef{PY@tok@ni}{\let\PY@bf=\textbf\def\PY@tc##1{\textcolor[rgb]{0.44,0.44,0.44}{##1}}}
\@namedef{PY@tok@na}{\def\PY@tc##1{\textcolor[rgb]{0.41,0.47,0.13}{##1}}}
\@namedef{PY@tok@nt}{\let\PY@bf=\textbf\def\PY@tc##1{\textcolor[rgb]{0.00,0.50,0.00}{##1}}}
\@namedef{PY@tok@nd}{\def\PY@tc##1{\textcolor[rgb]{0.67,0.13,1.00}{##1}}}
\@namedef{PY@tok@s}{\def\PY@tc##1{\textcolor[rgb]{0.73,0.13,0.13}{##1}}}
\@namedef{PY@tok@sd}{\let\PY@it=\textit\def\PY@tc##1{\textcolor[rgb]{0.73,0.13,0.13}{##1}}}
\@namedef{PY@tok@si}{\let\PY@bf=\textbf\def\PY@tc##1{\textcolor[rgb]{0.64,0.35,0.47}{##1}}}
\@namedef{PY@tok@se}{\let\PY@bf=\textbf\def\PY@tc##1{\textcolor[rgb]{0.67,0.36,0.12}{##1}}}
\@namedef{PY@tok@sr}{\def\PY@tc##1{\textcolor[rgb]{0.64,0.35,0.47}{##1}}}
\@namedef{PY@tok@ss}{\def\PY@tc##1{\textcolor[rgb]{0.10,0.09,0.49}{##1}}}
\@namedef{PY@tok@sx}{\def\PY@tc##1{\textcolor[rgb]{0.00,0.50,0.00}{##1}}}
\@namedef{PY@tok@m}{\def\PY@tc##1{\textcolor[rgb]{0.40,0.40,0.40}{##1}}}
\@namedef{PY@tok@gh}{\let\PY@bf=\textbf\def\PY@tc##1{\textcolor[rgb]{0.00,0.00,0.50}{##1}}}
\@namedef{PY@tok@gu}{\let\PY@bf=\textbf\def\PY@tc##1{\textcolor[rgb]{0.50,0.00,0.50}{##1}}}
\@namedef{PY@tok@gd}{\def\PY@tc##1{\textcolor[rgb]{0.63,0.00,0.00}{##1}}}
\@namedef{PY@tok@gi}{\def\PY@tc##1{\textcolor[rgb]{0.00,0.52,0.00}{##1}}}
\@namedef{PY@tok@gr}{\def\PY@tc##1{\textcolor[rgb]{0.89,0.00,0.00}{##1}}}
\@namedef{PY@tok@ge}{\let\PY@it=\textit}
\@namedef{PY@tok@gs}{\let\PY@bf=\textbf}
\@namedef{PY@tok@gp}{\let\PY@bf=\textbf\def\PY@tc##1{\textcolor[rgb]{0.00,0.00,0.50}{##1}}}
\@namedef{PY@tok@go}{\def\PY@tc##1{\textcolor[rgb]{0.44,0.44,0.44}{##1}}}
\@namedef{PY@tok@gt}{\def\PY@tc##1{\textcolor[rgb]{0.00,0.27,0.87}{##1}}}
\@namedef{PY@tok@err}{\def\PY@bc##1{{\setlength{\fboxsep}{\string -\fboxrule}\fcolorbox[rgb]{1.00,0.00,0.00}{1,1,1}{\strut ##1}}}}
\@namedef{PY@tok@kc}{\let\PY@bf=\textbf\def\PY@tc##1{\textcolor[rgb]{0.00,0.50,0.00}{##1}}}
\@namedef{PY@tok@kd}{\let\PY@bf=\textbf\def\PY@tc##1{\textcolor[rgb]{0.00,0.50,0.00}{##1}}}
\@namedef{PY@tok@kn}{\let\PY@bf=\textbf\def\PY@tc##1{\textcolor[rgb]{0.00,0.50,0.00}{##1}}}
\@namedef{PY@tok@kr}{\let\PY@bf=\textbf\def\PY@tc##1{\textcolor[rgb]{0.00,0.50,0.00}{##1}}}
\@namedef{PY@tok@bp}{\def\PY@tc##1{\textcolor[rgb]{0.00,0.50,0.00}{##1}}}
\@namedef{PY@tok@fm}{\def\PY@tc##1{\textcolor[rgb]{0.00,0.00,1.00}{##1}}}
\@namedef{PY@tok@vc}{\def\PY@tc##1{\textcolor[rgb]{0.10,0.09,0.49}{##1}}}
\@namedef{PY@tok@vg}{\def\PY@tc##1{\textcolor[rgb]{0.10,0.09,0.49}{##1}}}
\@namedef{PY@tok@vi}{\def\PY@tc##1{\textcolor[rgb]{0.10,0.09,0.49}{##1}}}
\@namedef{PY@tok@vm}{\def\PY@tc##1{\textcolor[rgb]{0.10,0.09,0.49}{##1}}}
\@namedef{PY@tok@sa}{\def\PY@tc##1{\textcolor[rgb]{0.73,0.13,0.13}{##1}}}
\@namedef{PY@tok@sb}{\def\PY@tc##1{\textcolor[rgb]{0.73,0.13,0.13}{##1}}}
\@namedef{PY@tok@sc}{\def\PY@tc##1{\textcolor[rgb]{0.73,0.13,0.13}{##1}}}
\@namedef{PY@tok@dl}{\def\PY@tc##1{\textcolor[rgb]{0.73,0.13,0.13}{##1}}}
\@namedef{PY@tok@s2}{\def\PY@tc##1{\textcolor[rgb]{0.73,0.13,0.13}{##1}}}
\@namedef{PY@tok@sh}{\def\PY@tc##1{\textcolor[rgb]{0.73,0.13,0.13}{##1}}}
\@namedef{PY@tok@s1}{\def\PY@tc##1{\textcolor[rgb]{0.73,0.13,0.13}{##1}}}
\@namedef{PY@tok@mb}{\def\PY@tc##1{\textcolor[rgb]{0.40,0.40,0.40}{##1}}}
\@namedef{PY@tok@mf}{\def\PY@tc##1{\textcolor[rgb]{0.40,0.40,0.40}{##1}}}
\@namedef{PY@tok@mh}{\def\PY@tc##1{\textcolor[rgb]{0.40,0.40,0.40}{##1}}}
\@namedef{PY@tok@mi}{\def\PY@tc##1{\textcolor[rgb]{0.40,0.40,0.40}{##1}}}
\@namedef{PY@tok@il}{\def\PY@tc##1{\textcolor[rgb]{0.40,0.40,0.40}{##1}}}
\@namedef{PY@tok@mo}{\def\PY@tc##1{\textcolor[rgb]{0.40,0.40,0.40}{##1}}}
\@namedef{PY@tok@ch}{\let\PY@it=\textit\def\PY@tc##1{\textcolor[rgb]{0.24,0.48,0.48}{##1}}}
\@namedef{PY@tok@cm}{\let\PY@it=\textit\def\PY@tc##1{\textcolor[rgb]{0.24,0.48,0.48}{##1}}}
\@namedef{PY@tok@cpf}{\let\PY@it=\textit\def\PY@tc##1{\textcolor[rgb]{0.24,0.48,0.48}{##1}}}
\@namedef{PY@tok@c1}{\let\PY@it=\textit\def\PY@tc##1{\textcolor[rgb]{0.24,0.48,0.48}{##1}}}
\@namedef{PY@tok@cs}{\let\PY@it=\textit\def\PY@tc##1{\textcolor[rgb]{0.24,0.48,0.48}{##1}}}

\def\PYZbs{\char`\\}
\def\PYZus{\char`\_}
\def\PYZob{\char`\{}
\def\PYZcb{\char`\}}
\def\PYZca{\char`\^}
\def\PYZam{\char`\&}
\def\PYZlt{\char`\<}
\def\PYZgt{\char`\>}
\def\PYZsh{\char`\#}
\def\PYZpc{\char`\%}
\def\PYZdl{\char`\$}
\def\PYZhy{\char`\-}
\def\PYZsq{\char`\'}
\def\PYZdq{\char`\"}
\def\PYZti{\char`\~}
% for compatibility with earlier versions
\def\PYZat{@}
\def\PYZlb{[}
\def\PYZrb{]}
\makeatother


    % For linebreaks inside Verbatim environment from package fancyvrb. 
    \makeatletter
        \newbox\Wrappedcontinuationbox 
        \newbox\Wrappedvisiblespacebox 
        \newcommand*\Wrappedvisiblespace {\textcolor{red}{\textvisiblespace}} 
        \newcommand*\Wrappedcontinuationsymbol {\textcolor{red}{\llap{\tiny$\m@th\hookrightarrow$}}} 
        \newcommand*\Wrappedcontinuationindent {3ex } 
        \newcommand*\Wrappedafterbreak {\kern\Wrappedcontinuationindent\copy\Wrappedcontinuationbox} 
        % Take advantage of the already applied Pygments mark-up to insert 
        % potential linebreaks for TeX processing. 
        %        {, <, #, %, $, ' and ": go to next line. 
        %        _, }, ^, &, >, - and ~: stay at end of broken line. 
        % Use of \textquotesingle for straight quote. 
        \newcommand*\Wrappedbreaksatspecials {% 
            \def\PYGZus{\discretionary{\char`\_}{\Wrappedafterbreak}{\char`\_}}% 
            \def\PYGZob{\discretionary{}{\Wrappedafterbreak\char`\{}{\char`\{}}% 
            \def\PYGZcb{\discretionary{\char`\}}{\Wrappedafterbreak}{\char`\}}}% 
            \def\PYGZca{\discretionary{\char`\^}{\Wrappedafterbreak}{\char`\^}}% 
            \def\PYGZam{\discretionary{\char`\&}{\Wrappedafterbreak}{\char`\&}}% 
            \def\PYGZlt{\discretionary{}{\Wrappedafterbreak\char`\<}{\char`\<}}% 
            \def\PYGZgt{\discretionary{\char`\>}{\Wrappedafterbreak}{\char`\>}}% 
            \def\PYGZsh{\discretionary{}{\Wrappedafterbreak\char`\#}{\char`\#}}% 
            \def\PYGZpc{\discretionary{}{\Wrappedafterbreak\char`\%}{\char`\%}}% 
            \def\PYGZdl{\discretionary{}{\Wrappedafterbreak\char`\$}{\char`\$}}% 
            \def\PYGZhy{\discretionary{\char`\-}{\Wrappedafterbreak}{\char`\-}}% 
            \def\PYGZsq{\discretionary{}{\Wrappedafterbreak\textquotesingle}{\textquotesingle}}% 
            \def\PYGZdq{\discretionary{}{\Wrappedafterbreak\char`\"}{\char`\"}}% 
            \def\PYGZti{\discretionary{\char`\~}{\Wrappedafterbreak}{\char`\~}}% 
        } 
        % Some characters . , ; ? ! / are not pygmentized. 
        % This macro makes them "active" and they will insert potential linebreaks 
        \newcommand*\Wrappedbreaksatpunct {% 
            \lccode`\~`\.\lowercase{\def~}{\discretionary{\hbox{\char`\.}}{\Wrappedafterbreak}{\hbox{\char`\.}}}% 
            \lccode`\~`\,\lowercase{\def~}{\discretionary{\hbox{\char`\,}}{\Wrappedafterbreak}{\hbox{\char`\,}}}% 
            \lccode`\~`\;\lowercase{\def~}{\discretionary{\hbox{\char`\;}}{\Wrappedafterbreak}{\hbox{\char`\;}}}% 
            \lccode`\~`\:\lowercase{\def~}{\discretionary{\hbox{\char`\:}}{\Wrappedafterbreak}{\hbox{\char`\:}}}% 
            \lccode`\~`\?\lowercase{\def~}{\discretionary{\hbox{\char`\?}}{\Wrappedafterbreak}{\hbox{\char`\?}}}% 
            \lccode`\~`\!\lowercase{\def~}{\discretionary{\hbox{\char`\!}}{\Wrappedafterbreak}{\hbox{\char`\!}}}% 
            \lccode`\~`\/\lowercase{\def~}{\discretionary{\hbox{\char`\/}}{\Wrappedafterbreak}{\hbox{\char`\/}}}% 
            \catcode`\.\active
            \catcode`\,\active 
            \catcode`\;\active
            \catcode`\:\active
            \catcode`\?\active
            \catcode`\!\active
            \catcode`\/\active 
            \lccode`\~`\~ 	
        }
    \makeatother

    \let\OriginalVerbatim=\Verbatim
    \makeatletter
    \renewcommand{\Verbatim}[1][1]{%
        %\parskip\z@skip
        \sbox\Wrappedcontinuationbox {\Wrappedcontinuationsymbol}%
        \sbox\Wrappedvisiblespacebox {\FV@SetupFont\Wrappedvisiblespace}%
        \def\FancyVerbFormatLine ##1{\hsize\linewidth
            \vtop{\raggedright\hyphenpenalty\z@\exhyphenpenalty\z@
                \doublehyphendemerits\z@\finalhyphendemerits\z@
                \strut ##1\strut}%
        }%
        % If the linebreak is at a space, the latter will be displayed as visible
        % space at end of first line, and a continuation symbol starts next line.
        % Stretch/shrink are however usually zero for typewriter font.
        \def\FV@Space {%
            \nobreak\hskip\z@ plus\fontdimen3\font minus\fontdimen4\font
            \discretionary{\copy\Wrappedvisiblespacebox}{\Wrappedafterbreak}
            {\kern\fontdimen2\font}%
        }%
        
        % Allow breaks at special characters using \PYG... macros.
        \Wrappedbreaksatspecials
        % Breaks at punctuation characters . , ; ? ! and / need catcode=\active 	
        \OriginalVerbatim[#1,codes*=\Wrappedbreaksatpunct]%
    }
    \makeatother

    % Exact colors from NB
    \definecolor{incolor}{HTML}{303F9F}
    \definecolor{outcolor}{HTML}{D84315}
    \definecolor{cellborder}{HTML}{CFCFCF}
    \definecolor{cellbackground}{HTML}{F7F7F7}
    
    % prompt
    \makeatletter
    \newcommand{\boxspacing}{\kern\kvtcb@left@rule\kern\kvtcb@boxsep}
    \makeatother
    \newcommand{\prompt}[4]{
        {\ttfamily\llap{{\color{#2}[#3]:\hspace{3pt}#4}}\vspace{-\baselineskip}}
    }
    

    
    % Prevent overflowing lines due to hard-to-break entities
    \sloppy 
    % Setup hyperref package
    \hypersetup{
      breaklinks=true,  % so long urls are correctly broken across lines
      colorlinks=true,
      urlcolor=urlcolor,
      linkcolor=linkcolor,
      citecolor=citecolor,
      }
    % Slightly bigger margins than the latex defaults
    
    \geometry{verbose,tmargin=1in,bmargin=1in,lmargin=1in,rmargin=1in}
    
    

\begin{document}
    
    \maketitle
    
    

    
    \hypertarget{plasma-studies}{%
\subsection{Plasma Studies}\label{plasma-studies}}

We perform various studies on the plasma system using the simulation.

    \begin{tcolorbox}[breakable, size=fbox, boxrule=1pt, pad at break*=1mm,colback=cellbackground, colframe=cellborder]
\prompt{In}{incolor}{1}{\boxspacing}
\begin{Verbatim}[commandchars=\\\{\}]
\PY{k+kn}{import} \PY{n+nn}{import\PYZus{}ipynb}
\PY{k+kn}{from} \PY{n+nn}{run} \PY{k+kn}{import} \PY{n}{Run}
\PY{k+kn}{from} \PY{n+nn}{constants} \PY{k+kn}{import} \PY{n}{Constants}
\PY{k+kn}{import} \PY{n+nn}{numpy} \PY{k}{as} \PY{n+nn}{np}
\end{Verbatim}
\end{tcolorbox}

    \begin{Verbatim}[commandchars=\\\{\}]
importing Jupyter notebook from run.ipynb
importing Jupyter notebook from batch.ipynb
importing Jupyter notebook from particle.ipynb
importing Jupyter notebook from field.ipynb
importing Jupyter notebook from constants.ipynb
    \end{Verbatim}

    \begin{tcolorbox}[breakable, size=fbox, boxrule=1pt, pad at break*=1mm,colback=cellbackground, colframe=cellborder]
\prompt{In}{incolor}{2}{\boxspacing}
\begin{Verbatim}[commandchars=\\\{\}]
\PY{n}{constants} \PY{o}{=} \PY{n}{Constants}\PY{p}{(}\PY{p}{)}
\end{Verbatim}
\end{tcolorbox}

    \hypertarget{section-1-checks}{%
\subsubsection{Section 1: Checks}\label{section-1-checks}}

Before we can do any studies confidently, we need to know if our program
works as we expect it to; or wheather there might be bugs causing some
erros. For this we perform a few checks, comparing calculations done by
hand against the results of the program.

As of now, we can check if the particle sampling and particle update is
performed as expected.

    \begin{tcolorbox}[breakable, size=fbox, boxrule=1pt, pad at break*=1mm,colback=cellbackground, colframe=cellborder]
\prompt{In}{incolor}{ }{\boxspacing}
\begin{Verbatim}[commandchars=\\\{\}]

\end{Verbatim}
\end{tcolorbox}

    \hypertarget{sampling}{%
\paragraph{1.1 Sampling}\label{sampling}}

So far we have used the Maxwellian distribution for sampling.

    \hypertarget{maxwellian-sampling}{%
\subparagraph{1.1.1 Maxwellian sampling}\label{maxwellian-sampling}}

For a Maxwellian distribution, we can calculate the average speeds of
the particles based on the input parameters like the Plasma temperature
and the mass of the species (assuming the same types of species for
now). Here we get the average speeds of the sampled particles; which we
will compare with a manual calculation.

    \begin{tcolorbox}[breakable, size=fbox, boxrule=1pt, pad at break*=1mm,colback=cellbackground, colframe=cellborder]
\prompt{In}{incolor}{3}{\boxspacing}
\begin{Verbatim}[commandchars=\\\{\}]
\PY{c+c1}{\PYZsh{} c1Maxwell means checking the Maxwellian sampling}
\PY{n}{c1Maxwell} \PY{o}{=} \PY{n}{Run}\PY{p}{(}\PY{p}{)}
\end{Verbatim}
\end{tcolorbox}

    \begin{tcolorbox}[breakable, size=fbox, boxrule=1pt, pad at break*=1mm,colback=cellbackground, colframe=cellborder]
\prompt{In}{incolor}{4}{\boxspacing}
\begin{Verbatim}[commandchars=\\\{\}]
\PY{c+c1}{\PYZsh{} Create 100 particles based on the data available in the files}
\PY{n}{c1Maxwell}\PY{o}{.}\PY{n}{create\PYZus{}batch\PYZus{}with\PYZus{}file\PYZus{}initialization}\PY{p}{(}\PY{l+s+s1}{\PYZsq{}}\PY{l+s+s1}{H+}\PY{l+s+s1}{\PYZsq{}}\PY{p}{,} \PY{n}{constants}\PY{o}{.}\PY{n}{constants}\PY{p}{[}\PY{l+s+s1}{\PYZsq{}}\PY{l+s+s1}{e}\PY{l+s+s1}{\PYZsq{}}\PY{p}{]}\PY{p}{[}\PY{l+m+mi}{0}\PY{p}{]}\PY{p}{,}\PYZbs{}
                                          \PY{n}{constants}\PY{o}{.}\PY{n}{constants}\PY{p}{[}\PY{l+s+s1}{\PYZsq{}}\PY{l+s+s1}{m\PYZus{}H}\PY{l+s+s1}{\PYZsq{}}\PY{p}{]}\PY{p}{[}\PY{l+m+mi}{0}\PY{p}{]} \PY{o}{*} \PY{n}{constants}\PY{o}{.}\PY{n}{constants}\PY{p}{[}\PY{l+s+s1}{\PYZsq{}}\PY{l+s+s1}{amu}\PY{l+s+s1}{\PYZsq{}}\PY{p}{]}\PY{p}{[}\PY{l+m+mi}{0}\PY{p}{]}\PY{p}{,} \PYZbs{}
                                          \PY{l+m+mi}{100}\PY{p}{,} \PY{l+m+mi}{100}\PY{p}{,} \PY{l+s+s1}{\PYZsq{}}\PY{l+s+s1}{H ions}\PY{l+s+s1}{\PYZsq{}}\PY{p}{,} \PY{n}{r\PYZus{}index}\PY{o}{=}\PY{l+m+mi}{0}\PY{p}{,} \PY{n}{v\PYZus{}index}\PY{o}{=}\PY{l+m+mi}{1}\PY{p}{)}
\end{Verbatim}
\end{tcolorbox}

    \begin{tcolorbox}[breakable, size=fbox, boxrule=1pt, pad at break*=1mm,colback=cellbackground, colframe=cellborder]
\prompt{In}{incolor}{5}{\boxspacing}
\begin{Verbatim}[commandchars=\\\{\}]
\PY{c+c1}{\PYZsh{} Take the 0th batch of particles}
\PY{n}{c1Maxwell\PYZus{}batch} \PY{o}{=} \PY{n}{c1Maxwell}\PY{o}{.}\PY{n}{batches}\PY{p}{[}\PY{l+m+mi}{0}\PY{p}{]}\PY{p}{[}\PY{l+s+s1}{\PYZsq{}}\PY{l+s+s1}{H ions}\PY{l+s+s1}{\PYZsq{}}\PY{p}{]}
\end{Verbatim}
\end{tcolorbox}

    \begin{tcolorbox}[breakable, size=fbox, boxrule=1pt, pad at break*=1mm,colback=cellbackground, colframe=cellborder]
\prompt{In}{incolor}{6}{\boxspacing}
\begin{Verbatim}[commandchars=\\\{\}]
\PY{c+c1}{\PYZsh{} Take the initial positions and velocities of the particles}
\PY{n}{c1Maxwell\PYZus{}positions} \PY{o}{=} \PY{p}{[}\PY{p}{]}
\PY{n}{c1Maxwell\PYZus{}velocities} \PY{o}{=} \PY{p}{[}\PY{p}{]}
\PY{k}{for} \PY{n}{particle} \PY{o+ow}{in} \PY{n}{c1Maxwell\PYZus{}batch}\PY{o}{.}\PY{n}{particles}\PY{p}{:}
    \PY{n}{c1Maxwell\PYZus{}positions}\PY{o}{.}\PY{n}{append}\PY{p}{(}\PY{n}{particle}\PY{o}{.}\PY{n}{r}\PY{p}{)}
    \PY{n}{c1Maxwell\PYZus{}velocities}\PY{o}{.}\PY{n}{append}\PY{p}{(}\PY{n}{particle}\PY{o}{.}\PY{n}{v}\PY{p}{)}
\end{Verbatim}
\end{tcolorbox}

    \begin{tcolorbox}[breakable, size=fbox, boxrule=1pt, pad at break*=1mm,colback=cellbackground, colframe=cellborder]
\prompt{In}{incolor}{7}{\boxspacing}
\begin{Verbatim}[commandchars=\\\{\}]
\PY{c+c1}{\PYZsh{} Let\PYZsq{}s look at the positions}
\PY{n}{c1Maxwell\PYZus{}positions}
\PY{c+c1}{\PYZsh{} They look fine, all particles are supposed to be initialized at [\PYZhy{}0.5, 0, 0]}
\PY{c+c1}{\PYZsh{} where a cube of sides 1m is assumed as the chamber and is described between coordinates (\PYZhy{}0.5, 0.5) for x, y and z coordinates}
\end{Verbatim}
\end{tcolorbox}

            \begin{tcolorbox}[breakable, size=fbox, boxrule=.5pt, pad at break*=1mm, opacityfill=0]
\prompt{Out}{outcolor}{7}{\boxspacing}
\begin{Verbatim}[commandchars=\\\{\}]
[array([-0.5,  0. ,  0. ]),
 array([-0.5,  0. ,  0. ]),
 array([-0.5,  0. ,  0. ]),
 array([-0.5,  0. ,  0. ]),
 array([-0.5,  0. ,  0. ]),
 array([-0.5,  0. ,  0. ]),
 array([-0.5,  0. ,  0. ]),
 array([-0.5,  0. ,  0. ]),
 array([-0.5,  0. ,  0. ]),
 array([-0.5,  0. ,  0. ]),
 array([-0.5,  0. ,  0. ]),
 array([-0.5,  0. ,  0. ]),
 array([-0.5,  0. ,  0. ]),
 array([-0.5,  0. ,  0. ]),
 array([-0.5,  0. ,  0. ]),
 array([-0.5,  0. ,  0. ]),
 array([-0.5,  0. ,  0. ]),
 array([-0.5,  0. ,  0. ]),
 array([-0.5,  0. ,  0. ]),
 array([-0.5,  0. ,  0. ]),
 array([-0.5,  0. ,  0. ]),
 array([-0.5,  0. ,  0. ]),
 array([-0.5,  0. ,  0. ]),
 array([-0.5,  0. ,  0. ]),
 array([-0.5,  0. ,  0. ]),
 array([-0.5,  0. ,  0. ]),
 array([-0.5,  0. ,  0. ]),
 array([-0.5,  0. ,  0. ]),
 array([-0.5,  0. ,  0. ]),
 array([-0.5,  0. ,  0. ]),
 array([-0.5,  0. ,  0. ]),
 array([-0.5,  0. ,  0. ]),
 array([-0.5,  0. ,  0. ]),
 array([-0.5,  0. ,  0. ]),
 array([-0.5,  0. ,  0. ]),
 array([-0.5,  0. ,  0. ]),
 array([-0.5,  0. ,  0. ]),
 array([-0.5,  0. ,  0. ]),
 array([-0.5,  0. ,  0. ]),
 array([-0.5,  0. ,  0. ]),
 array([-0.5,  0. ,  0. ]),
 array([-0.5,  0. ,  0. ]),
 array([-0.5,  0. ,  0. ]),
 array([-0.5,  0. ,  0. ]),
 array([-0.5,  0. ,  0. ]),
 array([-0.5,  0. ,  0. ]),
 array([-0.5,  0. ,  0. ]),
 array([-0.5,  0. ,  0. ]),
 array([-0.5,  0. ,  0. ]),
 array([-0.5,  0. ,  0. ]),
 array([-0.5,  0. ,  0. ]),
 array([-0.5,  0. ,  0. ]),
 array([-0.5,  0. ,  0. ]),
 array([-0.5,  0. ,  0. ]),
 array([-0.5,  0. ,  0. ]),
 array([-0.5,  0. ,  0. ]),
 array([-0.5,  0. ,  0. ]),
 array([-0.5,  0. ,  0. ]),
 array([-0.5,  0. ,  0. ]),
 array([-0.5,  0. ,  0. ]),
 array([-0.5,  0. ,  0. ]),
 array([-0.5,  0. ,  0. ]),
 array([-0.5,  0. ,  0. ]),
 array([-0.5,  0. ,  0. ]),
 array([-0.5,  0. ,  0. ]),
 array([-0.5,  0. ,  0. ]),
 array([-0.5,  0. ,  0. ]),
 array([-0.5,  0. ,  0. ]),
 array([-0.5,  0. ,  0. ]),
 array([-0.5,  0. ,  0. ]),
 array([-0.5,  0. ,  0. ]),
 array([-0.5,  0. ,  0. ]),
 array([-0.5,  0. ,  0. ]),
 array([-0.5,  0. ,  0. ]),
 array([-0.5,  0. ,  0. ]),
 array([-0.5,  0. ,  0. ]),
 array([-0.5,  0. ,  0. ]),
 array([-0.5,  0. ,  0. ]),
 array([-0.5,  0. ,  0. ]),
 array([-0.5,  0. ,  0. ]),
 array([-0.5,  0. ,  0. ]),
 array([-0.5,  0. ,  0. ]),
 array([-0.5,  0. ,  0. ]),
 array([-0.5,  0. ,  0. ]),
 array([-0.5,  0. ,  0. ]),
 array([-0.5,  0. ,  0. ]),
 array([-0.5,  0. ,  0. ]),
 array([-0.5,  0. ,  0. ]),
 array([-0.5,  0. ,  0. ]),
 array([-0.5,  0. ,  0. ]),
 array([-0.5,  0. ,  0. ]),
 array([-0.5,  0. ,  0. ]),
 array([-0.5,  0. ,  0. ]),
 array([-0.5,  0. ,  0. ]),
 array([-0.5,  0. ,  0. ]),
 array([-0.5,  0. ,  0. ]),
 array([-0.5,  0. ,  0. ]),
 array([-0.5,  0. ,  0. ]),
 array([-0.5,  0. ,  0. ]),
 array([-0.5,  0. ,  0. ])]
\end{Verbatim}
\end{tcolorbox}
        
    \begin{tcolorbox}[breakable, size=fbox, boxrule=1pt, pad at break*=1mm,colback=cellbackground, colframe=cellborder]
\prompt{In}{incolor}{8}{\boxspacing}
\begin{Verbatim}[commandchars=\\\{\}]
\PY{c+c1}{\PYZsh{} Let\PYZsq{}s now look at the velocities}
\PY{n}{c1Maxwell\PYZus{}velocities}
\PY{c+c1}{\PYZsh{} We need to check if they are really Maxwellian distributed}
\end{Verbatim}
\end{tcolorbox}

            \begin{tcolorbox}[breakable, size=fbox, boxrule=.5pt, pad at break*=1mm, opacityfill=0]
\prompt{Out}{outcolor}{8}{\boxspacing}
\begin{Verbatim}[commandchars=\\\{\}]
[array([ 1579.629935, -5723.201089, -2863.641674]),
 array([  897.689953, -5029.491038, -3292.544752]),
 array([ 386.944504, 1027.791639,  756.037653]),
 array([-9204.054924,  9615.130475, -1667.435373]),
 array([ -3641.215516, -10947.299598,  -3360.247147]),
 array([-5491.759943, -7127.660213,  2834.606515]),
 array([16625.53181 ,  7261.248008, 13182.501854]),
 array([ 3531.057922, -8670.854777,  5576.898277]),
 array([-6468.761439, -1081.00466 , 17040.23406 ]),
 array([-6782.695867, -1185.527177,  5481.859317]),
 array([ 2831.836461,  9814.523398, -4603.758126]),
 array([ 6601.297764, -5010.938691,  3131.004259]),
 array([   310.863591,  -1253.75245 , -11537.549212]),
 array([17607.190221, 10702.738406, 10743.593831]),
 array([  1841.536055,  -2739.779189, -12083.999033]),
 array([ 4245.610812, 11143.159359, -4244.35843 ]),
 array([ 6175.85488 , -5538.932063, -2785.536498]),
 array([-18786.399666,  -6481.17661 ,   8700.805047]),
 array([ -2630.875393, -12784.228172,   5763.276336]),
 array([14103.980493, -5259.190381, 11294.072721]),
 array([3218.123565, 2083.965365, 3701.724523]),
 array([  8110.7652  ,   1611.055445, -10797.61461 ]),
 array([ -5795.493516,  -6048.466929, -16270.769749]),
 array([ -5497.184733, -12480.106514,   1557.429411]),
 array([-13982.702604,   4446.07858 ,   8748.309843]),
 array([ 6562.348371,  2376.507555, 16756.176696]),
 array([ 1564.595443,  4923.296004, -1502.965034]),
 array([ 3999.538647,  5265.782508, 20981.06687 ]),
 array([2167.907498, 4715.441563, -383.229814]),
 array([-8141.437181,  -707.79037 , 18484.123017]),
 array([  4328.135274,  -2726.982622, -17110.480553]),
 array([14064.293936,  1939.197086,  8497.465692]),
 array([ 15755.484194,   1021.183689, -10923.123019]),
 array([ -4505.746111, -10715.356198,  -4957.862115]),
 array([  311.037312, 10372.211689, -7924.10112 ]),
 array([   194.746288, -16705.966313, -25158.665103]),
 array([3917.707815, 2788.029633, 3845.298211]),
 array([-6805.073646,  -114.745473, -2991.933822]),
 array([-20778.676767,   1204.886391,   1224.16076 ]),
 array([-1049.374644, 25399.499396,  3260.046569]),
 array([-1613.853665, 11450.244641,  6833.552532]),
 array([-4450.714515, 15520.095122, -4093.634965]),
 array([ -302.040006, -2774.294494,  8039.097495]),
 array([  837.443914, 19004.422265, -3366.240414]),
 array([-18089.011037,  -3702.184851,    973.063181]),
 array([ 1290.989261, -6265.270624, -6101.462157]),
 array([  627.872857,  7529.794087, -9586.239895]),
 array([-15877.458159,  -1685.890055,   2469.728849]),
 array([   401.980725,   4648.362411, -22877.706831]),
 array([ 13391.765702, -13467.286656,  -5302.578527]),
 array([  4424.274293,   1718.322218, -10218.543702]),
 array([-16491.662572,  -1376.577214,   1212.84242 ]),
 array([ 7398.623459, 12996.316882, -9925.234194]),
 array([-5142.450136, -4269.804713,  9158.1664  ]),
 array([ 2574.281443, 12535.307273,  1430.257433]),
 array([  9149.705078, -16458.827279,   3392.746971]),
 array([-7809.646458, -9975.384339,  6515.163693]),
 array([ 3364.849961,  9514.632099, 10146.688711]),
 array([11034.368466,  2228.723503, 10336.824708]),
 array([  3592.570787,   9359.729427, -15987.66503 ]),
 array([-9948.531042,  8687.897191, -1985.537473]),
 array([-21346.36499 ,    315.575762,  -4095.710895]),
 array([8745.421099, 1817.124211, -202.137507]),
 array([-4503.432576,  5444.488763, -4660.01537 ]),
 array([22320.899741,  -285.109853, 10270.464622]),
 array([11379.305971,  5524.826899,  3963.333405]),
 array([-4991.704923, -3821.720667,  2668.679899]),
 array([-1864.217875, -1091.525555, -5304.013859]),
 array([-11280.765472,  14952.883947,   6873.951826]),
 array([  7157.964588,  10566.45093 , -10696.666068]),
 array([-5803.460516,   174.639384,  7935.079774]),
 array([-7054.045335,  4371.138014, 14677.551923]),
 array([6675.742123, 2086.149166, 2812.089298]),
 array([ 1629.343137, 12814.296519, -1575.882811]),
 array([  1372.398103,   6908.124601, -11080.46534 ]),
 array([ 4077.427095,  9806.823708, -2952.864354]),
 array([20687.74124 , 10234.694055,  4265.535278]),
 array([-1161.868381, -9148.174165, -5257.7664  ]),
 array([ -9429.693013,   4349.497355, -17731.864223]),
 array([ 4360.301531, -2415.241844, -9765.025732]),
 array([12484.9937  , -5046.17275 ,  2094.336482]),
 array([  920.001379, -2826.506275, -6130.965348]),
 array([  7977.635576, -14613.882278,  -7575.019106]),
 array([ 2208.036015, -9531.481613,  5584.708069]),
 array([  -92.825592,  4472.404515, -2098.645854]),
 array([ -2929.734057, -12483.318971,   -383.726147]),
 array([ 7781.299771,  7546.658659, -7793.261556]),
 array([ 4995.961777, -9268.285396,  8484.176491]),
 array([11248.219314,  8449.902291, 10343.354335]),
 array([ -9901.37255 ,  -1143.945737, -11700.414233]),
 array([  1499.538688, -10332.224092,   1408.641439]),
 array([17961.404303, 15944.610445,  3105.683781]),
 array([17084.296062, -2164.739467, -2088.521291]),
 array([-11764.890976,  -4225.774814,   7203.046946]),
 array([ -4983.672944,   2659.037113, -14658.970414]),
 array([-2639.873802, -7685.606637,  5666.133006]),
 array([ -3893.19326 , -10739.510159,  12748.028668]),
 array([3349.164108,  527.511892, 1366.528714]),
 array([-11336.739018,  13658.143203,  13176.412033]),
 array([  -765.772707,  -3802.623207, -10641.798323])]
\end{Verbatim}
\end{tcolorbox}
        
    \begin{tcolorbox}[breakable, size=fbox, boxrule=1pt, pad at break*=1mm,colback=cellbackground, colframe=cellborder]
\prompt{In}{incolor}{9}{\boxspacing}
\begin{Verbatim}[commandchars=\\\{\}]
\PY{c+c1}{\PYZsh{} Get the speeds }
\PY{n}{c1Maxwell\PYZus{}speeds} \PY{o}{=} \PY{n}{np}\PY{o}{.}\PY{n}{sqrt}\PY{p}{(} \PYZbs{}
                           \PY{p}{[} \PY{p}{(}\PY{n}{c1Maxwell\PYZus{}velocities}\PY{p}{[}\PY{n}{i}\PY{p}{]}\PY{p}{[}\PY{l+m+mi}{0}\PY{p}{]} \PY{o}{*}\PY{o}{*} \PY{l+m+mi}{2}\PY{p}{)} \PY{o}{+} \PY{p}{(}\PY{n}{c1Maxwell\PYZus{}velocities}\PY{p}{[}\PY{n}{i}\PY{p}{]}\PY{p}{[}\PY{l+m+mi}{1}\PY{p}{]} \PY{o}{*}\PY{o}{*} \PY{l+m+mi}{2}\PY{p}{)} \PY{o}{+} \PYZbs{}
                            \PY{p}{(}\PY{n}{c1Maxwell\PYZus{}velocities}\PY{p}{[}\PY{n}{i}\PY{p}{]}\PY{p}{[}\PY{l+m+mi}{2}\PY{p}{]} \PY{o}{*}\PY{o}{*} \PY{l+m+mi}{2}\PY{p}{)} \PY{k}{for} \PY{n}{i} \PY{o+ow}{in} \PY{n+nb}{range}\PY{p}{(}\PY{n+nb}{len}\PY{p}{(}\PY{n}{c1Maxwell\PYZus{}velocities}\PY{p}{)}\PY{p}{)} \PY{p}{]} \PY{p}{)}
\end{Verbatim}
\end{tcolorbox}

    \begin{tcolorbox}[breakable, size=fbox, boxrule=1pt, pad at break*=1mm,colback=cellbackground, colframe=cellborder]
\prompt{In}{incolor}{10}{\boxspacing}
\begin{Verbatim}[commandchars=\\\{\}]
\PY{n}{c1Maxwell\PYZus{}speeds}
\end{Verbatim}
\end{tcolorbox}

            \begin{tcolorbox}[breakable, size=fbox, boxrule=.5pt, pad at break*=1mm, opacityfill=0]
\prompt{Out}{outcolor}{10}{\boxspacing}
\begin{Verbatim}[commandchars=\\\{\}]
array([ 6591.7148811 ,  6078.03243632,  1333.29465428, 13414.38413862,
       12016.36716363,  9433.87309001, 22425.70814333, 10897.42570538,
       18258.77932528,  8801.20556986, 11204.40788951,  8859.44842548,
       11609.63277617, 23237.61100611, 12526.79838091, 12657.40060792,
        8751.00939772, 21694.20362969, 14267.91328164, 18818.53947143,
        5329.35224641, 13600.31224062, 18300.5367547 , 13725.80361799,
       17082.63773428, 18151.62972763,  5380.12140539, 21998.40771087,
        5204.04430653, 20210.06606342, 17858.82789898, 16546.05005563,
       19198.76857515, 12637.28622839, 13056.45043995, 30200.75605565,
        6156.93609596,  7434.63932187, 20849.54985051, 25629.35153965,
       13431.68143601, 16656.53203751,  8509.70191245, 19318.40967731,
       18489.60099412,  8840.1419512 , 12206.06483031, 16156.59195053,
       23348.62590073, 19718.63427135, 11267.00801696, 16593.39886673,
       17948.653303  , 11337.90269399, 12876.58688088, 19134.28416282,
       14245.91971255, 14311.0352888 , 15283.1423778 , 18871.05012758,
       13356.46609518, 21738.0250732 ,  8934.49439917,  8463.98878223,
       24572.06333273, 13255.9544479 ,  6829.67929688,  5727.06725464,
       19952.33374001, 16652.47751145,  9832.40275289, 16861.10712789,
        7538.26223048, 13013.23791256, 13129.44683553, 11023.54801541,
       23471.82121652, 10615.23133073, 20548.87457705, 10963.63763323,
       13628.10230413,  6813.53625941, 18291.77790908, 11265.59048601,
        4941.18742377, 12828.24383664, 13350.49095434, 13521.908112  ,
       17461.62264403, 15370.27922475, 10535.07198209, 24217.51264119,
       17347.08015999, 14427.53332216, 15709.64315252,  9906.69199637,
       17117.42583262,  3655.48488503, 22106.20639955, 11326.7039619 ])
\end{Verbatim}
\end{tcolorbox}
        
    \begin{tcolorbox}[breakable, size=fbox, boxrule=1pt, pad at break*=1mm,colback=cellbackground, colframe=cellborder]
\prompt{In}{incolor}{11}{\boxspacing}
\begin{Verbatim}[commandchars=\\\{\}]
\PY{n}{c1Maxwell\PYZus{}meanspeed} \PY{o}{=} \PY{n}{np}\PY{o}{.}\PY{n}{sum}\PY{p}{(}\PY{n}{c1Maxwell\PYZus{}speeds}\PY{p}{)} \PY{o}{/} \PY{n}{c1Maxwell\PYZus{}speeds}\PY{o}{.}\PY{n}{size}
\PY{n}{c1Maxwell\PYZus{}meanspeed}
\end{Verbatim}
\end{tcolorbox}

            \begin{tcolorbox}[breakable, size=fbox, boxrule=.5pt, pad at break*=1mm, opacityfill=0]
\prompt{Out}{outcolor}{11}{\boxspacing}
\begin{Verbatim}[commandchars=\\\{\}]
14202.764572898674
\end{Verbatim}
\end{tcolorbox}
        
    \begin{tcolorbox}[breakable, size=fbox, boxrule=1pt, pad at break*=1mm,colback=cellbackground, colframe=cellborder]
\prompt{In}{incolor}{12}{\boxspacing}
\begin{Verbatim}[commandchars=\\\{\}]
\PY{n}{c1Maxwell\PYZus{}meanspeed\PYZus{}expected} \PY{o}{=} \PY{p}{(}\PY{l+m+mi}{2} \PY{o}{/} \PY{n}{np}\PY{o}{.}\PY{n}{sqrt}\PY{p}{(}\PY{n}{np}\PY{o}{.}\PY{n}{pi}\PY{p}{)} \PY{p}{)} \PY{o}{*} \PY{n}{np}\PY{o}{.}\PY{n}{sqrt}\PY{p}{(} \PY{p}{(}\PY{l+m+mi}{2} \PY{o}{*} \PY{n}{constants}\PY{o}{.}\PY{n}{constants}\PY{p}{[}\PY{l+s+s1}{\PYZsq{}}\PY{l+s+s1}{K}\PY{l+s+s1}{\PYZsq{}}\PY{p}{]}\PY{p}{[}\PY{l+m+mi}{0}\PY{p}{]} \PY{o}{*} \PY{n}{constants}\PY{o}{.}\PY{n}{constants}\PY{p}{[}\PY{l+s+s1}{\PYZsq{}}\PY{l+s+s1}{N\PYZus{}A}\PY{l+s+s1}{\PYZsq{}}\PY{p}{]}\PY{p}{[}\PY{l+m+mi}{0}\PY{p}{]} \PY{o}{*} \PY{l+m+mi}{10000}\PY{p}{)}\PY{o}{/} \PY{p}{(} \PY{n}{constants}\PY{o}{.}\PY{n}{constants}\PY{p}{[}\PY{l+s+s1}{\PYZsq{}}\PY{l+s+s1}{m\PYZus{}H}\PY{l+s+s1}{\PYZsq{}}\PY{p}{]}\PY{p}{[}\PY{l+m+mi}{0}\PY{p}{]} \PY{o}{*} \PY{l+m+mi}{10}\PY{o}{*}\PY{o}{*}\PY{p}{(}\PY{o}{\PYZhy{}}\PY{l+m+mi}{3}\PY{p}{)}\PY{p}{)} \PY{p}{)}
\PY{n}{c1Maxwell\PYZus{}meanspeed\PYZus{}expected}
\end{Verbatim}
\end{tcolorbox}

            \begin{tcolorbox}[breakable, size=fbox, boxrule=.5pt, pad at break*=1mm, opacityfill=0]
\prompt{Out}{outcolor}{12}{\boxspacing}
\begin{Verbatim}[commandchars=\\\{\}]
14492.952993825973
\end{Verbatim}
\end{tcolorbox}
        
    \begin{tcolorbox}[breakable, size=fbox, boxrule=1pt, pad at break*=1mm,colback=cellbackground, colframe=cellborder]
\prompt{In}{incolor}{41}{\boxspacing}
\begin{Verbatim}[commandchars=\\\{\}]
\PY{p}{(}\PY{n}{c1Maxwell\PYZus{}meanspeed\PYZus{}expected} \PY{o}{\PYZhy{}} \PY{n}{c1Maxwell\PYZus{}meanspeed}\PY{p}{)} \PY{o}{*} \PY{l+m+mi}{100}\PY{o}{/} \PY{n}{c1Maxwell\PYZus{}meanspeed}
\PY{c+c1}{\PYZsh{} We see a maximum of 2.04 \PYZpc{} discrepency}
\end{Verbatim}
\end{tcolorbox}

            \begin{tcolorbox}[breakable, size=fbox, boxrule=.5pt, pad at break*=1mm, opacityfill=0]
\prompt{Out}{outcolor}{41}{\boxspacing}
\begin{Verbatim}[commandchars=\\\{\}]
2.0431826454479785
\end{Verbatim}
\end{tcolorbox}
        
    \hypertarget{we-see-that-the-numbers-are-close}{%
\paragraph{We see that the numbers are
close}\label{we-see-that-the-numbers-are-close}}

We were using Hydrogen atom samples. from the second (1th) available
velocity sampled file. Now let's use Hydrogen gas samples, from the
third (2nd) available velocity sampled file; to see if the numbers are
still close.

    \begin{tcolorbox}[breakable, size=fbox, boxrule=1pt, pad at break*=1mm,colback=cellbackground, colframe=cellborder]
\prompt{In}{incolor}{13}{\boxspacing}
\begin{Verbatim}[commandchars=\\\{\}]
\PY{c+c1}{\PYZsh{} Create 100 particles based on the sampled velocities of H2 gas}
\PY{c+c1}{\PYZsh{} We consider just for this test case that a Hydrogen molecule is a particle of the plasma}
\PY{c+c1}{\PYZsh{} We create a new batch on the same Run instance}
\PY{n}{c1Maxwell}\PY{o}{.}\PY{n}{create\PYZus{}batch\PYZus{}with\PYZus{}file\PYZus{}initialization}\PY{p}{(}\PY{l+s+s1}{\PYZsq{}}\PY{l+s+s1}{H2}\PY{l+s+s1}{\PYZsq{}}\PY{p}{,} \PY{n}{constants}\PY{o}{.}\PY{n}{constants}\PY{p}{[}\PY{l+s+s1}{\PYZsq{}}\PY{l+s+s1}{e}\PY{l+s+s1}{\PYZsq{}}\PY{p}{]}\PY{p}{[}\PY{l+m+mi}{0}\PY{p}{]}\PY{p}{,}\PYZbs{}
                                          \PY{l+m+mi}{2} \PY{o}{*} \PY{n}{constants}\PY{o}{.}\PY{n}{constants}\PY{p}{[}\PY{l+s+s1}{\PYZsq{}}\PY{l+s+s1}{m\PYZus{}H}\PY{l+s+s1}{\PYZsq{}}\PY{p}{]}\PY{p}{[}\PY{l+m+mi}{0}\PY{p}{]} \PY{o}{*} \PY{n}{constants}\PY{o}{.}\PY{n}{constants}\PY{p}{[}\PY{l+s+s1}{\PYZsq{}}\PY{l+s+s1}{amu}\PY{l+s+s1}{\PYZsq{}}\PY{p}{]}\PY{p}{[}\PY{l+m+mi}{0}\PY{p}{]}\PY{p}{,} \PYZbs{}
                                          \PY{l+m+mi}{100}\PY{p}{,} \PY{l+m+mi}{100}\PY{p}{,} \PY{l+s+s1}{\PYZsq{}}\PY{l+s+s1}{H2 gas}\PY{l+s+s1}{\PYZsq{}}\PY{p}{,} \PY{n}{r\PYZus{}index}\PY{o}{=}\PY{l+m+mi}{0}\PY{p}{,} \PY{n}{v\PYZus{}index}\PY{o}{=}\PY{l+m+mi}{2}\PY{p}{)}
\end{Verbatim}
\end{tcolorbox}

    \begin{tcolorbox}[breakable, size=fbox, boxrule=1pt, pad at break*=1mm,colback=cellbackground, colframe=cellborder]
\prompt{In}{incolor}{38}{\boxspacing}
\begin{Verbatim}[commandchars=\\\{\}]
\PY{c+c1}{\PYZsh{} Do the same as before for the second batch}
\PY{n}{c1Maxwell\PYZus{}batch2} \PY{o}{=} \PY{n}{c1Maxwell}\PY{o}{.}\PY{n}{batches}\PY{p}{[}\PY{l+m+mi}{1}\PY{p}{]}\PY{p}{[}\PY{l+s+s1}{\PYZsq{}}\PY{l+s+s1}{H2 gas}\PY{l+s+s1}{\PYZsq{}}\PY{p}{]}
\PY{n}{c1Maxwell\PYZus{}positions\PYZus{}batch2} \PY{o}{=} \PY{p}{[}\PY{p}{]}
\PY{n}{c1Maxwell\PYZus{}velocities\PYZus{}batch2} \PY{o}{=} \PY{p}{[}\PY{p}{]}
\PY{k}{for} \PY{n}{particle} \PY{o+ow}{in} \PY{n}{c1Maxwell\PYZus{}batch2}\PY{o}{.}\PY{n}{particles}\PY{p}{:}
    \PY{n}{c1Maxwell\PYZus{}positions\PYZus{}batch2}\PY{o}{.}\PY{n}{append}\PY{p}{(}\PY{n}{particle}\PY{o}{.}\PY{n}{r}\PY{p}{)}
    \PY{n}{c1Maxwell\PYZus{}velocities\PYZus{}batch2}\PY{o}{.}\PY{n}{append}\PY{p}{(}\PY{n}{particle}\PY{o}{.}\PY{n}{v}\PY{p}{)}
\PY{n}{c1Maxwell\PYZus{}speeds\PYZus{}batch2} \PY{o}{=} \PY{n}{np}\PY{o}{.}\PY{n}{sqrt}\PY{p}{(} \PYZbs{}
                           \PY{p}{[} \PY{p}{(}\PY{n}{c1Maxwell\PYZus{}velocities\PYZus{}batch2}\PY{p}{[}\PY{n}{i}\PY{p}{]}\PY{p}{[}\PY{l+m+mi}{0}\PY{p}{]} \PY{o}{*}\PY{o}{*} \PY{l+m+mi}{2}\PY{p}{)} \PY{o}{+} \PY{p}{(}\PY{n}{c1Maxwell\PYZus{}velocities\PYZus{}batch2}\PY{p}{[}\PY{n}{i}\PY{p}{]}\PY{p}{[}\PY{l+m+mi}{1}\PY{p}{]} \PY{o}{*}\PY{o}{*} \PY{l+m+mi}{2}\PY{p}{)} \PY{o}{+} \PYZbs{}
                            \PY{p}{(}\PY{n}{c1Maxwell\PYZus{}velocities\PYZus{}batch2}\PY{p}{[}\PY{n}{i}\PY{p}{]}\PY{p}{[}\PY{l+m+mi}{2}\PY{p}{]} \PY{o}{*}\PY{o}{*} \PY{l+m+mi}{2}\PY{p}{)} \PY{k}{for} \PY{n}{i} \PY{o+ow}{in} \PY{n+nb}{range}\PY{p}{(}\PY{n+nb}{len}\PY{p}{(}\PY{n}{c1Maxwell\PYZus{}velocities\PYZus{}batch2}\PY{p}{)}\PY{p}{)} \PY{p}{]} \PY{p}{)}
\PY{n}{c1Maxwell\PYZus{}meanspeed\PYZus{}batch2} \PY{o}{=} \PY{n}{np}\PY{o}{.}\PY{n}{sum}\PY{p}{(}\PY{n}{c1Maxwell\PYZus{}speeds\PYZus{}batch2}\PY{p}{)} \PY{o}{/} \PY{n}{c1Maxwell\PYZus{}speeds\PYZus{}batch2}\PY{o}{.}\PY{n}{size}
\PY{n}{c1Maxwell\PYZus{}meanspeed\PYZus{}batch2}
\end{Verbatim}
\end{tcolorbox}

            \begin{tcolorbox}[breakable, size=fbox, boxrule=.5pt, pad at break*=1mm, opacityfill=0]
\prompt{Out}{outcolor}{38}{\boxspacing}
\begin{Verbatim}[commandchars=\\\{\}]
10149.754879907316
\end{Verbatim}
\end{tcolorbox}
        
    \begin{tcolorbox}[breakable, size=fbox, boxrule=1pt, pad at break*=1mm,colback=cellbackground, colframe=cellborder]
\prompt{In}{incolor}{39}{\boxspacing}
\begin{Verbatim}[commandchars=\\\{\}]
\PY{c+c1}{\PYZsh{} We multiply the mass of Hydrogen by 2 to get the numbers for hydrogen molecule}
\PY{n}{c1Maxwell\PYZus{}meanspeed\PYZus{}expected\PYZus{}batch2} \PY{o}{=} \PY{p}{(}\PY{l+m+mi}{2} \PY{o}{/} \PY{n}{np}\PY{o}{.}\PY{n}{sqrt}\PY{p}{(}\PY{n}{np}\PY{o}{.}\PY{n}{pi}\PY{p}{)} \PY{p}{)} \PY{o}{*} \PY{n}{np}\PY{o}{.}\PY{n}{sqrt}\PY{p}{(} \PY{p}{(}\PY{l+m+mi}{2} \PY{o}{*} \PY{n}{constants}\PY{o}{.}\PY{n}{constants}\PY{p}{[}\PY{l+s+s1}{\PYZsq{}}\PY{l+s+s1}{K}\PY{l+s+s1}{\PYZsq{}}\PY{p}{]}\PY{p}{[}\PY{l+m+mi}{0}\PY{p}{]} \PY{o}{*} \PY{n}{constants}\PY{o}{.}\PY{n}{constants}\PY{p}{[}\PY{l+s+s1}{\PYZsq{}}\PY{l+s+s1}{N\PYZus{}A}\PY{l+s+s1}{\PYZsq{}}\PY{p}{]}\PY{p}{[}\PY{l+m+mi}{0}\PY{p}{]} \PY{o}{*} \PY{l+m+mi}{10000}\PY{p}{)}\PY{o}{/} \PY{p}{(} \PY{l+m+mi}{2} \PY{o}{*} \PY{n}{constants}\PY{o}{.}\PY{n}{constants}\PY{p}{[}\PY{l+s+s1}{\PYZsq{}}\PY{l+s+s1}{m\PYZus{}H}\PY{l+s+s1}{\PYZsq{}}\PY{p}{]}\PY{p}{[}\PY{l+m+mi}{0}\PY{p}{]} \PY{o}{*} \PY{l+m+mi}{10}\PY{o}{*}\PY{o}{*}\PY{p}{(}\PY{o}{\PYZhy{}}\PY{l+m+mi}{3}\PY{p}{)}\PY{p}{)} \PY{p}{)}
\PY{n}{c1Maxwell\PYZus{}meanspeed\PYZus{}expected\PYZus{}batch2}
\end{Verbatim}
\end{tcolorbox}

            \begin{tcolorbox}[breakable, size=fbox, boxrule=.5pt, pad at break*=1mm, opacityfill=0]
\prompt{Out}{outcolor}{39}{\boxspacing}
\begin{Verbatim}[commandchars=\\\{\}]
10248.06534135222
\end{Verbatim}
\end{tcolorbox}
        
    \begin{tcolorbox}[breakable, size=fbox, boxrule=1pt, pad at break*=1mm,colback=cellbackground, colframe=cellborder]
\prompt{In}{incolor}{42}{\boxspacing}
\begin{Verbatim}[commandchars=\\\{\}]
\PY{p}{(}\PY{n}{c1Maxwell\PYZus{}meanspeed\PYZus{}expected\PYZus{}batch2} \PY{o}{\PYZhy{}} \PY{n}{c1Maxwell\PYZus{}meanspeed\PYZus{}batch2}\PY{p}{)} \PY{o}{*} \PY{l+m+mi}{100}\PY{o}{/} \PY{n}{c1Maxwell\PYZus{}meanspeed\PYZus{}batch2}
\PY{c+c1}{\PYZsh{} We see a maximum of 0.97\PYZpc{} discrepency}
\end{Verbatim}
\end{tcolorbox}

            \begin{tcolorbox}[breakable, size=fbox, boxrule=.5pt, pad at break*=1mm, opacityfill=0]
\prompt{Out}{outcolor}{42}{\boxspacing}
\begin{Verbatim}[commandchars=\\\{\}]
0.9685993662716086
\end{Verbatim}
\end{tcolorbox}
        
    \hypertarget{again-we-see-that-the-numbers-are-close-enough}{%
\paragraph{Again we see that the numbers are close
enough}\label{again-we-see-that-the-numbers-are-close-enough}}

For a sample of large enough number of particles, we would expect the
match to be better. However, for now we take this as an indication that
our program is working as we expect it to.

    \begin{tcolorbox}[breakable, size=fbox, boxrule=1pt, pad at break*=1mm,colback=cellbackground, colframe=cellborder]
\prompt{In}{incolor}{ }{\boxspacing}
\begin{Verbatim}[commandchars=\\\{\}]

\end{Verbatim}
\end{tcolorbox}

    \hypertarget{update}{%
\paragraph{1.2 Update}\label{update}}

Here we check if our particle update works as expected.

    \hypertarget{boris-update}{%
\subparagraph{1.2.1 Boris Update}\label{boris-update}}

So far we use the Boris Algorithm to update individual particles in the
Electric and Magnetic fields, here we set up a simple configuration, and
observe the result of update of a single particle. We compare it to
calculations done by hand.

    \begin{tcolorbox}[breakable, size=fbox, boxrule=1pt, pad at break*=1mm,colback=cellbackground, colframe=cellborder]
\prompt{In}{incolor}{14}{\boxspacing}
\begin{Verbatim}[commandchars=\\\{\}]
\PY{c+c1}{\PYZsh{} c2Boris means checking the Boris update}
\PY{n}{c2Boris} \PY{o}{=} \PY{n}{Run}\PY{p}{(}\PY{p}{)}
\end{Verbatim}
\end{tcolorbox}

    \begin{tcolorbox}[breakable, size=fbox, boxrule=1pt, pad at break*=1mm,colback=cellbackground, colframe=cellborder]
\prompt{In}{incolor}{15}{\boxspacing}
\begin{Verbatim}[commandchars=\\\{\}]
\PY{c+c1}{\PYZsh{}Create 10 particles}
\PY{n}{c2Boris}\PY{o}{.}\PY{n}{create\PYZus{}batch\PYZus{}with\PYZus{}file\PYZus{}initialization}\PY{p}{(}\PY{l+s+s1}{\PYZsq{}}\PY{l+s+s1}{H+}\PY{l+s+s1}{\PYZsq{}}\PY{p}{,} \PY{n}{constants}\PY{o}{.}\PY{n}{constants}\PY{p}{[}\PY{l+s+s1}{\PYZsq{}}\PY{l+s+s1}{e}\PY{l+s+s1}{\PYZsq{}}\PY{p}{]}\PY{p}{[}\PY{l+m+mi}{0}\PY{p}{]}\PY{p}{,}\PYZbs{}
                                          \PY{n}{constants}\PY{o}{.}\PY{n}{constants}\PY{p}{[}\PY{l+s+s1}{\PYZsq{}}\PY{l+s+s1}{m\PYZus{}H}\PY{l+s+s1}{\PYZsq{}}\PY{p}{]}\PY{p}{[}\PY{l+m+mi}{0}\PY{p}{]} \PY{o}{*} \PY{n}{constants}\PY{o}{.}\PY{n}{constants}\PY{p}{[}\PY{l+s+s1}{\PYZsq{}}\PY{l+s+s1}{amu}\PY{l+s+s1}{\PYZsq{}}\PY{p}{]}\PY{p}{[}\PY{l+m+mi}{0}\PY{p}{]}\PY{p}{,} \PYZbs{}
                                          \PY{l+m+mi}{100}\PY{p}{,} \PY{l+m+mi}{10}\PY{p}{,} \PY{l+s+s1}{\PYZsq{}}\PY{l+s+s1}{H ions}\PY{l+s+s1}{\PYZsq{}}\PY{p}{,} \PY{n}{r\PYZus{}index}\PY{o}{=}\PY{l+m+mi}{0}\PY{p}{,} \PY{n}{v\PYZus{}index}\PY{o}{=}\PY{l+m+mi}{1}\PY{p}{)}
\end{Verbatim}
\end{tcolorbox}

    \begin{tcolorbox}[breakable, size=fbox, boxrule=1pt, pad at break*=1mm,colback=cellbackground, colframe=cellborder]
\prompt{In}{incolor}{16}{\boxspacing}
\begin{Verbatim}[commandchars=\\\{\}]
\PY{n}{c2Boris\PYZus{}index\PYZus{}update} \PY{o}{=} \PY{l+m+mi}{0} \PY{c+c1}{\PYZsh{} Update the first batch in this Run instance run\PYZus{}Boris\PYZus{}check}
\PY{n}{c2Boris\PYZus{}particle\PYZus{}track\PYZus{}indices} \PY{o}{=} \PY{p}{[}\PY{n}{i} \PY{k}{for} \PY{n}{i} \PY{o+ow}{in} \PY{n+nb}{range}\PY{p}{(}\PY{l+m+mi}{10}\PY{p}{)}\PY{p}{]} \PY{c+c1}{\PYZsh{} Track all 10 particles}
\PY{n}{c2Boris\PYZus{}dT} \PY{o}{=} \PY{l+m+mi}{10}\PY{o}{*}\PY{o}{*}\PY{p}{(}\PY{o}{\PYZhy{}}\PY{l+m+mi}{6}\PY{p}{)} \PY{c+c1}{\PYZsh{} 1 microseconds}
\PY{n}{c2Boris\PYZus{}stepT} \PY{o}{=} \PY{l+m+mi}{10}\PY{o}{*}\PY{o}{*}\PY{p}{(}\PY{o}{\PYZhy{}}\PY{l+m+mi}{7}\PY{p}{)} \PY{c+c1}{\PYZsh{} 0.1 microseconds time step}
\PY{n}{c2Boris\PYZus{}E0} \PY{o}{=} \PY{l+m+mi}{1000} \PY{c+c1}{\PYZsh{} say 1000 Volts (voltage) per meter (size of chamber) }
\PY{n}{c2Boris\PYZus{}Edirn} \PY{o}{=} \PY{p}{[}\PY{l+m+mi}{1}\PY{p}{,}\PY{l+m+mi}{0}\PY{p}{,}\PY{l+m+mi}{0}\PY{p}{]} \PY{c+c1}{\PYZsh{}in the x\PYZhy{}direction [1,0,0]}
\PY{n}{c2Boris\PYZus{}B0} \PY{o}{=} \PY{l+m+mi}{10} \PY{o}{*} \PY{p}{(}\PY{l+m+mi}{10}\PY{o}{*}\PY{o}{*}\PY{p}{(}\PY{o}{\PYZhy{}}\PY{l+m+mi}{3}\PY{p}{)}\PY{p}{)} \PY{c+c1}{\PYZsh{} Meant to say 10 mT }
\PY{n}{c2Boris\PYZus{}Bdirn} \PY{o}{=} \PY{p}{[}\PY{l+m+mi}{0}\PY{p}{,}\PY{l+m+mi}{1}\PY{p}{,}\PY{l+m+mi}{0}\PY{p}{]} \PY{c+c1}{\PYZsh{}in the y\PYZhy{}direction [0,1,0]}
\PY{n}{c2Boris\PYZus{}argsE} \PY{o}{=} \PY{p}{[}\PY{n}{element} \PY{o}{*} \PY{n}{c2Boris\PYZus{}E0} \PY{k}{for} \PY{n}{element} \PY{o+ow}{in} \PY{n}{c2Boris\PYZus{}Edirn}\PY{p}{]} \PY{c+c1}{\PYZsh{} currently the uniform\PYZus{}E\PYZus{}field configuration is used}
\PY{n}{c2Boris\PYZus{}argsB} \PY{o}{=} \PY{p}{[}\PY{n}{element} \PY{o}{*} \PY{n}{c2Boris\PYZus{}B0} \PY{k}{for} \PY{n}{element} \PY{o+ow}{in} \PY{n}{c2Boris\PYZus{}Bdirn}\PY{p}{]}\PY{c+c1}{\PYZsh{} currently the uniform\PYZus{}B\PYZus{}field configuration is used}
\end{Verbatim}
\end{tcolorbox}

    \begin{tcolorbox}[breakable, size=fbox, boxrule=1pt, pad at break*=1mm,colback=cellbackground, colframe=cellborder]
\prompt{In}{incolor}{17}{\boxspacing}
\begin{Verbatim}[commandchars=\\\{\}]
\PY{n}{c2Boris\PYZus{}positions\PYZus{}and\PYZus{}velocities} \PY{o}{=} \PY{n}{c2Boris}\PY{o}{.}\PY{n}{update\PYZus{}batch\PYZus{}with\PYZus{}unchanging\PYZus{}fields}\PY{p}{(}\PY{n}{c2Boris\PYZus{}index\PYZus{}update}\PY{p}{,} \PYZbs{}
                                                                               \PY{n}{c2Boris\PYZus{}dT}\PY{p}{,} \PY{n}{c2Boris\PYZus{}stepT}\PY{p}{,} \PYZbs{}
                                                                               \PY{n}{c2Boris\PYZus{}argsE}\PY{p}{,} \PY{n}{c2Boris\PYZus{}argsB}\PY{p}{,} \PYZbs{}
                                                                               \PY{n}{c2Boris\PYZus{}particle\PYZus{}track\PYZus{}indices}\PY{p}{)}
\end{Verbatim}
\end{tcolorbox}

    \begin{tcolorbox}[breakable, size=fbox, boxrule=1pt, pad at break*=1mm,colback=cellbackground, colframe=cellborder]
\prompt{In}{incolor}{18}{\boxspacing}
\begin{Verbatim}[commandchars=\\\{\}]
\PY{c+c1}{\PYZsh{}Let\PYZsq{}s inspect the positions and velocities of the particles at index 1 and 7}
\PY{n}{c2Boris\PYZus{}p1} \PY{o}{=} \PY{n}{c2Boris\PYZus{}positions\PYZus{}and\PYZus{}velocities}\PY{p}{[}\PY{l+m+mi}{1}\PY{p}{]}
\PY{n}{c2Boris\PYZus{}p7} \PY{o}{=} \PY{n}{c2Boris\PYZus{}positions\PYZus{}and\PYZus{}velocities}\PY{p}{[}\PY{l+m+mi}{7}\PY{p}{]}
\end{Verbatim}
\end{tcolorbox}

    \begin{tcolorbox}[breakable, size=fbox, boxrule=1pt, pad at break*=1mm,colback=cellbackground, colframe=cellborder]
\prompt{In}{incolor}{19}{\boxspacing}
\begin{Verbatim}[commandchars=\\\{\}]
\PY{c+c1}{\PYZsh{}Let\PYZsq{}s look at particle 1\PYZsq{}s positions and velocities}
\PY{n}{c2Boris\PYZus{}p1}
\end{Verbatim}
\end{tcolorbox}

            \begin{tcolorbox}[breakable, size=fbox, boxrule=.5pt, pad at break*=1mm, opacityfill=0]
\prompt{Out}{outcolor}{19}{\boxspacing}
\begin{Verbatim}[commandchars=\\\{\}]
[(0,
  array([-4.98921519e-01, -5.02949104e-04, -2.74850639e-04]),
  array([10784.80849437, -5029.491038  , -2748.50639353])),
 (1,
  array([-4.96859534e-01, -1.00589821e-03, -4.00658364e-04]),
  array([20619.85191499, -5029.491038  , -1258.07724484])),
 (2,
  array([-4.93828311e-01, -1.50884731e-03, -2.83282555e-04]),
  array([30312.2320901 , -5029.491038  ,  1173.75808561])),
 (3,
  array([-4.89851127e-01, -2.01179642e-03,  1.70051838e-04]),
  array([39771.83801957, -5029.491038  ,  4533.34393251])),
 (4,
  array([-0.48496014, -0.00251475,  0.00104989]),
  array([48909.86581774, -5029.491038  ,  8798.39924387])),
 (5,
  array([-0.47919618, -0.00301769,  0.00244371]),
  array([57639.6443318 , -5029.491038  , 13938.14269746])),
 (6,
  array([-0.47260843, -0.00352064,  0.00443506]),
  array([65877.44878403, -5029.491038  , 19913.49684507])),
 (7,
  array([-0.4652541 , -0.00402359,  0.00710279]),
  array([73543.29487349, -5029.491038  , 26677.37013834])),
 (8,
  array([-0.45719793, -0.00452654,  0.01052029]),
  array([80561.70588222, -5029.491038  , 34175.01496554])),
 (9,
  array([-0.44851169, -0.00502949,  0.01475474]),
  array([86862.44551045, -5029.491038  , 42344.45911554]))]
\end{Verbatim}
\end{tcolorbox}
        
    \begin{tcolorbox}[breakable, size=fbox, boxrule=1pt, pad at break*=1mm,colback=cellbackground, colframe=cellborder]
\prompt{In}{incolor}{20}{\boxspacing}
\begin{Verbatim}[commandchars=\\\{\}]
\PY{c+c1}{\PYZsh{}Let\PYZsq{}s take the positions and velocities of the particle 1 after the 3rd and the 4th time steps.}
\PY{n}{c2Boris\PYZus{}p134\PYZus{}p} \PY{o}{=} \PY{p}{[}\PY{n}{c2Boris\PYZus{}p1}\PY{p}{[}\PY{l+m+mi}{3}\PY{p}{]}\PY{p}{[}\PY{l+m+mi}{1}\PY{p}{]}\PY{p}{,} \PY{n}{c2Boris\PYZus{}p1}\PY{p}{[}\PY{l+m+mi}{4}\PY{p}{]}\PY{p}{[}\PY{l+m+mi}{1}\PY{p}{]}\PY{p}{]}
\PY{n}{c2Boris\PYZus{}p134\PYZus{}v} \PY{o}{=} \PY{p}{[}\PY{n}{c2Boris\PYZus{}p1}\PY{p}{[}\PY{l+m+mi}{3}\PY{p}{]}\PY{p}{[}\PY{l+m+mi}{2}\PY{p}{]}\PY{p}{,} \PY{n}{c2Boris\PYZus{}p1}\PY{p}{[}\PY{l+m+mi}{4}\PY{p}{]}\PY{p}{[}\PY{l+m+mi}{2}\PY{p}{]}\PY{p}{]}
\end{Verbatim}
\end{tcolorbox}

    \begin{tcolorbox}[breakable, size=fbox, boxrule=1pt, pad at break*=1mm,colback=cellbackground, colframe=cellborder]
\prompt{In}{incolor}{21}{\boxspacing}
\begin{Verbatim}[commandchars=\\\{\}]
\PY{c+c1}{\PYZsh{}Likewise for particle 7}
\PY{n}{c2Boris\PYZus{}p734\PYZus{}p} \PY{o}{=} \PY{p}{[}\PY{n}{c2Boris\PYZus{}p1}\PY{p}{[}\PY{l+m+mi}{3}\PY{p}{]}\PY{p}{[}\PY{l+m+mi}{1}\PY{p}{]}\PY{p}{,} \PY{n}{c2Boris\PYZus{}p1}\PY{p}{[}\PY{l+m+mi}{4}\PY{p}{]}\PY{p}{[}\PY{l+m+mi}{1}\PY{p}{]}\PY{p}{]}
\PY{n}{c2Boris\PYZus{}p734\PYZus{}v} \PY{o}{=} \PY{p}{[}\PY{n}{c2Boris\PYZus{}p1}\PY{p}{[}\PY{l+m+mi}{3}\PY{p}{]}\PY{p}{[}\PY{l+m+mi}{2}\PY{p}{]}\PY{p}{,} \PY{n}{c2Boris\PYZus{}p1}\PY{p}{[}\PY{l+m+mi}{4}\PY{p}{]}\PY{p}{[}\PY{l+m+mi}{2}\PY{p}{]}\PY{p}{]}
\end{Verbatim}
\end{tcolorbox}

    \begin{tcolorbox}[breakable, size=fbox, boxrule=1pt, pad at break*=1mm,colback=cellbackground, colframe=cellborder]
\prompt{In}{incolor}{22}{\boxspacing}
\begin{Verbatim}[commandchars=\\\{\}]
\PY{c+c1}{\PYZsh{}Let\PYZsq{}s look at the positions to check if they match up against the positions and velocities from above}
\PY{n}{c2Boris\PYZus{}p134\PYZus{}p}
\end{Verbatim}
\end{tcolorbox}

            \begin{tcolorbox}[breakable, size=fbox, boxrule=.5pt, pad at break*=1mm, opacityfill=0]
\prompt{Out}{outcolor}{22}{\boxspacing}
\begin{Verbatim}[commandchars=\\\{\}]
[array([-4.89851127e-01, -2.01179642e-03,  1.70051838e-04]),
 array([-0.48496014, -0.00251475,  0.00104989])]
\end{Verbatim}
\end{tcolorbox}
        
    \begin{tcolorbox}[breakable, size=fbox, boxrule=1pt, pad at break*=1mm,colback=cellbackground, colframe=cellborder]
\prompt{In}{incolor}{23}{\boxspacing}
\begin{Verbatim}[commandchars=\\\{\}]
\PY{c+c1}{\PYZsh{}Let\PYZsq{}s look at the velocities to check if they match up against the positions and velocities from above}
\PY{n}{c2Boris\PYZus{}p134\PYZus{}v}
\end{Verbatim}
\end{tcolorbox}

            \begin{tcolorbox}[breakable, size=fbox, boxrule=.5pt, pad at break*=1mm, opacityfill=0]
\prompt{Out}{outcolor}{23}{\boxspacing}
\begin{Verbatim}[commandchars=\\\{\}]
[array([39771.83801957, -5029.491038  ,  4533.34393251]),
 array([48909.86581774, -5029.491038  ,  8798.39924387])]
\end{Verbatim}
\end{tcolorbox}
        
    \begin{tcolorbox}[breakable, size=fbox, boxrule=1pt, pad at break*=1mm,colback=cellbackground, colframe=cellborder]
\prompt{In}{incolor}{24}{\boxspacing}
\begin{Verbatim}[commandchars=\\\{\}]
\PY{n+nb}{print}\PY{p}{(}\PY{l+s+sa}{f}\PY{l+s+s1}{\PYZsq{}}\PY{l+s+s1}{The velocity changed from }\PY{l+s+se}{\PYZbs{}n}\PY{l+s+s1}{ }\PY{l+s+si}{\PYZob{}}\PY{p}{[}\PY{n}{c2Boris\PYZus{}p134\PYZus{}v}\PY{p}{[}\PY{l+m+mi}{0}\PY{p}{]}\PY{p}{[}\PY{l+m+mi}{0}\PY{p}{]}\PY{p}{,} \PY{n}{c2Boris\PYZus{}p134\PYZus{}v}\PY{p}{[}\PY{l+m+mi}{0}\PY{p}{]}\PY{p}{[}\PY{l+m+mi}{1}\PY{p}{]}\PY{p}{,} \PY{n}{c2Boris\PYZus{}p134\PYZus{}v}\PY{p}{[}\PY{l+m+mi}{0}\PY{p}{]}\PY{p}{[}\PY{l+m+mi}{2}\PY{p}{]}\PY{p}{]}\PY{l+s+si}{\PYZcb{}}\PY{l+s+s1}{ }\PY{l+s+se}{\PYZbs{}n}\PY{l+s+s1}{ to }\PY{l+s+se}{\PYZbs{}n}\PY{l+s+s1}{ }\PY{l+s+si}{\PYZob{}}\PY{p}{[}\PY{n}{c2Boris\PYZus{}p134\PYZus{}v}\PY{p}{[}\PY{l+m+mi}{1}\PY{p}{]}\PY{p}{[}\PY{l+m+mi}{0}\PY{p}{]}\PY{p}{,} \PY{n}{c2Boris\PYZus{}p134\PYZus{}v}\PY{p}{[}\PY{l+m+mi}{1}\PY{p}{]}\PY{p}{[}\PY{l+m+mi}{1}\PY{p}{]}\PY{p}{,} \PY{n}{c2Boris\PYZus{}p134\PYZus{}v}\PY{p}{[}\PY{l+m+mi}{1}\PY{p}{]}\PY{p}{[}\PY{l+m+mi}{2}\PY{p}{]}\PY{p}{]}\PY{l+s+si}{\PYZcb{}}\PY{l+s+s1}{ }\PY{l+s+se}{\PYZbs{}n}\PY{l+s+s1}{\PYZsq{}}\PY{p}{)}
\PY{n+nb}{print}\PY{p}{(}\PY{l+s+sa}{f}\PY{l+s+s1}{\PYZsq{}}\PY{l+s+s1}{The position changed from }\PY{l+s+se}{\PYZbs{}n}\PY{l+s+s1}{ }\PY{l+s+si}{\PYZob{}}\PY{p}{[}\PY{n}{c2Boris\PYZus{}p134\PYZus{}p}\PY{p}{[}\PY{l+m+mi}{0}\PY{p}{]}\PY{p}{[}\PY{l+m+mi}{0}\PY{p}{]}\PY{p}{,} \PY{n}{c2Boris\PYZus{}p134\PYZus{}p}\PY{p}{[}\PY{l+m+mi}{0}\PY{p}{]}\PY{p}{[}\PY{l+m+mi}{1}\PY{p}{]}\PY{p}{,} \PY{n}{c2Boris\PYZus{}p134\PYZus{}p}\PY{p}{[}\PY{l+m+mi}{0}\PY{p}{]}\PY{p}{[}\PY{l+m+mi}{2}\PY{p}{]}\PY{p}{]}\PY{l+s+si}{\PYZcb{}}\PY{l+s+s1}{ }\PY{l+s+se}{\PYZbs{}n}\PY{l+s+s1}{ to }\PY{l+s+se}{\PYZbs{}n}\PY{l+s+s1}{ }\PY{l+s+si}{\PYZob{}}\PY{p}{[}\PY{n}{c2Boris\PYZus{}p134\PYZus{}p}\PY{p}{[}\PY{l+m+mi}{1}\PY{p}{]}\PY{p}{[}\PY{l+m+mi}{0}\PY{p}{]}\PY{p}{,} \PY{n}{c2Boris\PYZus{}p134\PYZus{}p}\PY{p}{[}\PY{l+m+mi}{1}\PY{p}{]}\PY{p}{[}\PY{l+m+mi}{1}\PY{p}{]}\PY{p}{,} \PY{n}{c2Boris\PYZus{}p134\PYZus{}p}\PY{p}{[}\PY{l+m+mi}{1}\PY{p}{]}\PY{p}{[}\PY{l+m+mi}{2}\PY{p}{]}\PY{p}{]}\PY{l+s+si}{\PYZcb{}}\PY{l+s+s1}{ }\PY{l+s+se}{\PYZbs{}n}\PY{l+s+s1}{\PYZsq{}}\PY{p}{)}
\end{Verbatim}
\end{tcolorbox}

    \begin{Verbatim}[commandchars=\\\{\}]
The velocity changed from
 [39771.83801956555, -5029.491038, 4533.343932509505]
 to
 [48909.86581773698, -5029.491038, 8798.399243869582]

The position changed from
 [-0.48985112694809785, -0.0020117964152, 0.00017005183797547688]
 to
 [-0.4849601403663242, -0.0025147455189999997, 0.001049891762362435]

    \end{Verbatim}

    \begin{tcolorbox}[breakable, size=fbox, boxrule=1pt, pad at break*=1mm,colback=cellbackground, colframe=cellborder]
\prompt{In}{incolor}{25}{\boxspacing}
\begin{Verbatim}[commandchars=\\\{\}]
\PY{c+c1}{\PYZsh{}Similary for particle 7}
\PY{n}{c2Boris\PYZus{}p734\PYZus{}p} \PY{o}{=} \PY{p}{[}\PY{n}{c2Boris\PYZus{}p7}\PY{p}{[}\PY{l+m+mi}{3}\PY{p}{]}\PY{p}{[}\PY{l+m+mi}{1}\PY{p}{]}\PY{p}{,} \PY{n}{c2Boris\PYZus{}p7}\PY{p}{[}\PY{l+m+mi}{4}\PY{p}{]}\PY{p}{[}\PY{l+m+mi}{1}\PY{p}{]}\PY{p}{]}
\PY{n}{c2Boris\PYZus{}p734\PYZus{}v} \PY{o}{=} \PY{p}{[}\PY{n}{c2Boris\PYZus{}p7}\PY{p}{[}\PY{l+m+mi}{3}\PY{p}{]}\PY{p}{[}\PY{l+m+mi}{2}\PY{p}{]}\PY{p}{,} \PY{n}{c2Boris\PYZus{}p7}\PY{p}{[}\PY{l+m+mi}{4}\PY{p}{]}\PY{p}{[}\PY{l+m+mi}{2}\PY{p}{]}\PY{p}{]}
\PY{n+nb}{print}\PY{p}{(}\PY{l+s+sa}{f}\PY{l+s+s1}{\PYZsq{}}\PY{l+s+s1}{The velocity changed from }\PY{l+s+se}{\PYZbs{}n}\PY{l+s+s1}{ }\PY{l+s+si}{\PYZob{}}\PY{p}{[}\PY{n}{c2Boris\PYZus{}p734\PYZus{}v}\PY{p}{[}\PY{l+m+mi}{0}\PY{p}{]}\PY{p}{[}\PY{l+m+mi}{0}\PY{p}{]}\PY{p}{,} \PY{n}{c2Boris\PYZus{}p734\PYZus{}v}\PY{p}{[}\PY{l+m+mi}{0}\PY{p}{]}\PY{p}{[}\PY{l+m+mi}{1}\PY{p}{]}\PY{p}{,} \PY{n}{c2Boris\PYZus{}p734\PYZus{}v}\PY{p}{[}\PY{l+m+mi}{0}\PY{p}{]}\PY{p}{[}\PY{l+m+mi}{2}\PY{p}{]}\PY{p}{]}\PY{l+s+si}{\PYZcb{}}\PY{l+s+s1}{ }\PY{l+s+se}{\PYZbs{}n}\PY{l+s+s1}{ to }\PY{l+s+se}{\PYZbs{}n}\PY{l+s+s1}{ }\PY{l+s+si}{\PYZob{}}\PY{p}{[}\PY{n}{c2Boris\PYZus{}p734\PYZus{}v}\PY{p}{[}\PY{l+m+mi}{1}\PY{p}{]}\PY{p}{[}\PY{l+m+mi}{0}\PY{p}{]}\PY{p}{,} \PY{n}{c2Boris\PYZus{}p734\PYZus{}v}\PY{p}{[}\PY{l+m+mi}{1}\PY{p}{]}\PY{p}{[}\PY{l+m+mi}{1}\PY{p}{]}\PY{p}{,} \PY{n}{c2Boris\PYZus{}p734\PYZus{}v}\PY{p}{[}\PY{l+m+mi}{1}\PY{p}{]}\PY{p}{[}\PY{l+m+mi}{2}\PY{p}{]}\PY{p}{]}\PY{l+s+si}{\PYZcb{}}\PY{l+s+s1}{ }\PY{l+s+se}{\PYZbs{}n}\PY{l+s+s1}{\PYZsq{}}\PY{p}{)}
\PY{n+nb}{print}\PY{p}{(}\PY{l+s+sa}{f}\PY{l+s+s1}{\PYZsq{}}\PY{l+s+s1}{The position changed from }\PY{l+s+se}{\PYZbs{}n}\PY{l+s+s1}{ }\PY{l+s+si}{\PYZob{}}\PY{p}{[}\PY{n}{c2Boris\PYZus{}p734\PYZus{}p}\PY{p}{[}\PY{l+m+mi}{0}\PY{p}{]}\PY{p}{[}\PY{l+m+mi}{0}\PY{p}{]}\PY{p}{,} \PY{n}{c2Boris\PYZus{}p734\PYZus{}p}\PY{p}{[}\PY{l+m+mi}{0}\PY{p}{]}\PY{p}{[}\PY{l+m+mi}{1}\PY{p}{]}\PY{p}{,} \PY{n}{c2Boris\PYZus{}p734\PYZus{}p}\PY{p}{[}\PY{l+m+mi}{0}\PY{p}{]}\PY{p}{[}\PY{l+m+mi}{2}\PY{p}{]}\PY{p}{]}\PY{l+s+si}{\PYZcb{}}\PY{l+s+s1}{ }\PY{l+s+se}{\PYZbs{}n}\PY{l+s+s1}{ to }\PY{l+s+se}{\PYZbs{}n}\PY{l+s+s1}{ }\PY{l+s+si}{\PYZob{}}\PY{p}{[}\PY{n}{c2Boris\PYZus{}p734\PYZus{}p}\PY{p}{[}\PY{l+m+mi}{1}\PY{p}{]}\PY{p}{[}\PY{l+m+mi}{0}\PY{p}{]}\PY{p}{,} \PY{n}{c2Boris\PYZus{}p734\PYZus{}p}\PY{p}{[}\PY{l+m+mi}{1}\PY{p}{]}\PY{p}{[}\PY{l+m+mi}{1}\PY{p}{]}\PY{p}{,} \PY{n}{c2Boris\PYZus{}p734\PYZus{}p}\PY{p}{[}\PY{l+m+mi}{1}\PY{p}{]}\PY{p}{[}\PY{l+m+mi}{2}\PY{p}{]}\PY{p}{]}\PY{l+s+si}{\PYZcb{}}\PY{l+s+s1}{ }\PY{l+s+se}{\PYZbs{}n}\PY{l+s+s1}{\PYZsq{}}\PY{p}{)}
\end{Verbatim}
\end{tcolorbox}

    \begin{Verbatim}[commandchars=\\\{\}]
The velocity changed from
 [38895.85871094631, -8670.854777, 13914.969423620274]
 to
 [47135.881299118584, -8670.854777, 18096.176367366777]

The position changed from
 [-0.4896669752432719, -0.0034683419108, 0.0038875496903932644]
 to
 [-0.48495338711336006, -0.0043354273885, 0.0056971673271299424]

    \end{Verbatim}

    \begin{tcolorbox}[breakable, size=fbox, boxrule=1pt, pad at break*=1mm,colback=cellbackground, colframe=cellborder]
\prompt{In}{incolor}{26}{\boxspacing}
\begin{Verbatim}[commandchars=\\\{\}]
\PY{n}{c2Boris\PYZus{}qp} \PY{o}{=} \PY{p}{(}\PY{n}{constants}\PY{o}{.}\PY{n}{constants}\PY{p}{[}\PY{l+s+s1}{\PYZsq{}}\PY{l+s+s1}{e}\PY{l+s+s1}{\PYZsq{}}\PY{p}{]}\PY{p}{[}\PY{l+m+mi}{0}\PY{p}{]} \PY{o}{*} \PY{l+m+mi}{10}\PY{o}{*}\PY{o}{*}\PY{p}{(}\PY{o}{\PYZhy{}}\PY{l+m+mi}{7}\PY{p}{)}\PY{p}{)} \PY{o}{/} \PY{p}{(}\PY{l+m+mi}{2} \PY{o}{*} \PY{n}{constants}\PY{o}{.}\PY{n}{constants}\PY{p}{[}\PY{l+s+s1}{\PYZsq{}}\PY{l+s+s1}{m\PYZus{}H}\PY{l+s+s1}{\PYZsq{}}\PY{p}{]}\PY{p}{[}\PY{l+m+mi}{0}\PY{p}{]} \PY{o}{*} \PY{n}{constants}\PY{o}{.}\PY{n}{constants}\PY{p}{[}\PY{l+s+s1}{\PYZsq{}}\PY{l+s+s1}{amu}\PY{l+s+s1}{\PYZsq{}}\PY{p}{]}\PY{p}{[}\PY{l+m+mi}{0}\PY{p}{]}\PY{p}{)}
\end{Verbatim}
\end{tcolorbox}

    \begin{tcolorbox}[breakable, size=fbox, boxrule=1pt, pad at break*=1mm,colback=cellbackground, colframe=cellborder]
\prompt{In}{incolor}{27}{\boxspacing}
\begin{Verbatim}[commandchars=\\\{\}]
\PY{n}{c2Boris\PYZus{}qp} 
\end{Verbatim}
\end{tcolorbox}

            \begin{tcolorbox}[breakable, size=fbox, boxrule=.5pt, pad at break*=1mm, opacityfill=0]
\prompt{Out}{outcolor}{27}{\boxspacing}
\begin{Verbatim}[commandchars=\\\{\}]
4.78597877761177
\end{Verbatim}
\end{tcolorbox}
        
    \begin{tcolorbox}[breakable, size=fbox, boxrule=1pt, pad at break*=1mm,colback=cellbackground, colframe=cellborder]
\prompt{In}{incolor}{28}{\boxspacing}
\begin{Verbatim}[commandchars=\\\{\}]
\PY{c+c1}{\PYZsh{} The velocity after update 4, doing a manual update.}
\PY{n}{c2Boris\PYZus{}vminus\PYZus{}p1} \PY{o}{=} \PY{n}{np}\PY{o}{.}\PY{n}{add}\PY{p}{(}\PY{n}{c2Boris\PYZus{}p134\PYZus{}v}\PY{p}{[}\PY{l+m+mi}{0}\PY{p}{]}\PY{p}{,} \PY{n}{c2Boris\PYZus{}qp} \PY{o}{*} \PY{n}{np}\PY{o}{.}\PY{n}{array}\PY{p}{(}\PY{n}{c2Boris\PYZus{}argsE}\PY{p}{)}\PY{p}{)}
\PY{n}{c2Boris\PYZus{}vplus\PYZus{}p1} \PY{o}{=} \PY{n}{np}\PY{o}{.}\PY{n}{add}\PY{p}{(}\PY{n}{c2Boris\PYZus{}vminus\PYZus{}p1}\PY{p}{,} \PY{l+m+mi}{2} \PY{o}{*} \PY{n}{c2Boris\PYZus{}qp} \PY{o}{*} \PY{n}{np}\PY{o}{.}\PY{n}{cross}\PY{p}{(}\PY{n}{c2Boris\PYZus{}vminus\PYZus{}p1}\PY{p}{,} \PY{n}{c2Boris\PYZus{}argsB}\PY{p}{)}\PY{p}{)}
\PY{n}{c2Boris\PYZus{}vnew\PYZus{}p1} \PY{o}{=} \PY{n}{np}\PY{o}{.}\PY{n}{add}\PY{p}{(}\PY{n}{c2Boris\PYZus{}vplus\PYZus{}p1}\PY{p}{,} \PY{n}{c2Boris\PYZus{}qp} \PY{o}{*} \PY{n}{np}\PY{o}{.}\PY{n}{array}\PY{p}{(}\PY{n}{c2Boris\PYZus{}argsE}\PY{p}{)}\PY{p}{)}
\end{Verbatim}
\end{tcolorbox}

    \begin{tcolorbox}[breakable, size=fbox, boxrule=1pt, pad at break*=1mm,colback=cellbackground, colframe=cellborder]
\prompt{In}{incolor}{29}{\boxspacing}
\begin{Verbatim}[commandchars=\\\{\}]
\PY{n}{c2Boris\PYZus{}vnew\PYZus{}p1}
\PY{c+c1}{\PYZsh{}This is indeed the same as the velocity of particle 1 after update 4.}
\PY{n}{np}\PY{o}{.}\PY{n}{isclose}\PY{p}{(}\PY{n}{c2Boris\PYZus{}vnew\PYZus{}p1}\PY{p}{,} \PY{n}{c2Boris\PYZus{}p134\PYZus{}v}\PY{p}{[}\PY{l+m+mi}{1}\PY{p}{]}\PY{p}{)} \PY{c+c1}{\PYZsh{}Indeed the values are close (i.e. the same)}
\end{Verbatim}
\end{tcolorbox}

            \begin{tcolorbox}[breakable, size=fbox, boxrule=.5pt, pad at break*=1mm, opacityfill=0]
\prompt{Out}{outcolor}{29}{\boxspacing}
\begin{Verbatim}[commandchars=\\\{\}]
array([ True,  True,  True])
\end{Verbatim}
\end{tcolorbox}
        
    \begin{tcolorbox}[breakable, size=fbox, boxrule=1pt, pad at break*=1mm,colback=cellbackground, colframe=cellborder]
\prompt{In}{incolor}{30}{\boxspacing}
\begin{Verbatim}[commandchars=\\\{\}]
\PY{c+c1}{\PYZsh{}Similarly for position of the particle 1 after update 4}
\PY{n}{c2Boris\PYZus{}pnew\PYZus{}p1} \PY{o}{=} \PY{n}{np}\PY{o}{.}\PY{n}{add}\PY{p}{(}\PY{n}{c2Boris\PYZus{}p134\PYZus{}p}\PY{p}{[}\PY{l+m+mi}{0}\PY{p}{]}\PY{p}{,} \PY{n}{c2Boris\PYZus{}stepT} \PY{o}{*} \PY{n}{c2Boris\PYZus{}vnew\PYZus{}p1}\PY{p}{)}
\end{Verbatim}
\end{tcolorbox}

    \begin{tcolorbox}[breakable, size=fbox, boxrule=1pt, pad at break*=1mm,colback=cellbackground, colframe=cellborder]
\prompt{In}{incolor}{31}{\boxspacing}
\begin{Verbatim}[commandchars=\\\{\}]
\PY{n}{c2Boris\PYZus{}pnew\PYZus{}p1}
\PY{c+c1}{\PYZsh{}This is indeed the same as the position of particle 1 after update 4.}
\PY{n}{np}\PY{o}{.}\PY{n}{isclose}\PY{p}{(}\PY{n}{c2Boris\PYZus{}pnew\PYZus{}p1}\PY{p}{,} \PY{n}{c2Boris\PYZus{}p134\PYZus{}p}\PY{p}{[}\PY{l+m+mi}{1}\PY{p}{]}\PY{p}{)} \PY{c+c1}{\PYZsh{}Indeed th values are close (i.e. the same)}
\end{Verbatim}
\end{tcolorbox}

            \begin{tcolorbox}[breakable, size=fbox, boxrule=.5pt, pad at break*=1mm, opacityfill=0]
\prompt{Out}{outcolor}{31}{\boxspacing}
\begin{Verbatim}[commandchars=\\\{\}]
array([ True,  True,  True])
\end{Verbatim}
\end{tcolorbox}
        
    \begin{tcolorbox}[breakable, size=fbox, boxrule=1pt, pad at break*=1mm,colback=cellbackground, colframe=cellborder]
\prompt{In}{incolor}{32}{\boxspacing}
\begin{Verbatim}[commandchars=\\\{\}]
\PY{n}{c2Boris\PYZus{}p134\PYZus{}p}\PY{p}{[}\PY{l+m+mi}{0}\PY{p}{]}
\end{Verbatim}
\end{tcolorbox}

            \begin{tcolorbox}[breakable, size=fbox, boxrule=.5pt, pad at break*=1mm, opacityfill=0]
\prompt{Out}{outcolor}{32}{\boxspacing}
\begin{Verbatim}[commandchars=\\\{\}]
array([-4.89851127e-01, -2.01179642e-03,  1.70051838e-04])
\end{Verbatim}
\end{tcolorbox}
        
    \begin{tcolorbox}[breakable, size=fbox, boxrule=1pt, pad at break*=1mm,colback=cellbackground, colframe=cellborder]
\prompt{In}{incolor}{33}{\boxspacing}
\begin{Verbatim}[commandchars=\\\{\}]
\PY{n}{c2Boris\PYZus{}p134\PYZus{}v}\PY{p}{[}\PY{l+m+mi}{0}\PY{p}{]}
\end{Verbatim}
\end{tcolorbox}

            \begin{tcolorbox}[breakable, size=fbox, boxrule=.5pt, pad at break*=1mm, opacityfill=0]
\prompt{Out}{outcolor}{33}{\boxspacing}
\begin{Verbatim}[commandchars=\\\{\}]
array([39771.83801957, -5029.491038  ,  4533.34393251])
\end{Verbatim}
\end{tcolorbox}
        
    \begin{tcolorbox}[breakable, size=fbox, boxrule=1pt, pad at break*=1mm,colback=cellbackground, colframe=cellborder]
\prompt{In}{incolor}{ }{\boxspacing}
\begin{Verbatim}[commandchars=\\\{\}]

\end{Verbatim}
\end{tcolorbox}

    \begin{tcolorbox}[breakable, size=fbox, boxrule=1pt, pad at break*=1mm,colback=cellbackground, colframe=cellborder]
\prompt{In}{incolor}{34}{\boxspacing}
\begin{Verbatim}[commandchars=\\\{\}]
\PY{c+c1}{\PYZsh{}Similarly for particle 7}
\PY{n}{c2Boris\PYZus{}vminus\PYZus{}p7} \PY{o}{=} \PY{n}{np}\PY{o}{.}\PY{n}{add}\PY{p}{(}\PY{n}{c2Boris\PYZus{}p734\PYZus{}v}\PY{p}{[}\PY{l+m+mi}{0}\PY{p}{]}\PY{p}{,} \PY{n}{c2Boris\PYZus{}qp} \PY{o}{*} \PY{n}{np}\PY{o}{.}\PY{n}{array}\PY{p}{(}\PY{n}{c2Boris\PYZus{}argsE}\PY{p}{)}\PY{p}{)}
\PY{n}{c2Boris\PYZus{}vplus\PYZus{}p7} \PY{o}{=} \PY{n}{np}\PY{o}{.}\PY{n}{add}\PY{p}{(}\PY{n}{c2Boris\PYZus{}vminus\PYZus{}p7}\PY{p}{,} \PY{l+m+mi}{2} \PY{o}{*} \PY{n}{c2Boris\PYZus{}qp} \PY{o}{*} \PY{n}{np}\PY{o}{.}\PY{n}{cross}\PY{p}{(}\PY{n}{c2Boris\PYZus{}vminus\PYZus{}p7}\PY{p}{,} \PY{n}{c2Boris\PYZus{}argsB}\PY{p}{)}\PY{p}{)}
\PY{n}{c2Boris\PYZus{}vnew\PYZus{}p7} \PY{o}{=} \PY{n}{np}\PY{o}{.}\PY{n}{add}\PY{p}{(}\PY{n}{c2Boris\PYZus{}vplus\PYZus{}p7}\PY{p}{,} \PY{n}{c2Boris\PYZus{}qp} \PY{o}{*} \PY{n}{np}\PY{o}{.}\PY{n}{array}\PY{p}{(}\PY{n}{c2Boris\PYZus{}argsE}\PY{p}{)}\PY{p}{)}
\end{Verbatim}
\end{tcolorbox}

    \begin{tcolorbox}[breakable, size=fbox, boxrule=1pt, pad at break*=1mm,colback=cellbackground, colframe=cellborder]
\prompt{In}{incolor}{35}{\boxspacing}
\begin{Verbatim}[commandchars=\\\{\}]
\PY{n}{np}\PY{o}{.}\PY{n}{isclose}\PY{p}{(}\PY{n}{c2Boris\PYZus{}vnew\PYZus{}p7}\PY{p}{,} \PY{n}{c2Boris\PYZus{}p734\PYZus{}v}\PY{p}{[}\PY{l+m+mi}{1}\PY{p}{]}\PY{p}{)}
\end{Verbatim}
\end{tcolorbox}

            \begin{tcolorbox}[breakable, size=fbox, boxrule=.5pt, pad at break*=1mm, opacityfill=0]
\prompt{Out}{outcolor}{35}{\boxspacing}
\begin{Verbatim}[commandchars=\\\{\}]
array([ True,  True,  True])
\end{Verbatim}
\end{tcolorbox}
        
    \begin{tcolorbox}[breakable, size=fbox, boxrule=1pt, pad at break*=1mm,colback=cellbackground, colframe=cellborder]
\prompt{In}{incolor}{36}{\boxspacing}
\begin{Verbatim}[commandchars=\\\{\}]
\PY{n}{c2Boris\PYZus{}pnew\PYZus{}p7} \PY{o}{=} \PY{n}{np}\PY{o}{.}\PY{n}{add}\PY{p}{(}\PY{n}{c2Boris\PYZus{}p734\PYZus{}p}\PY{p}{[}\PY{l+m+mi}{0}\PY{p}{]}\PY{p}{,} \PY{n}{c2Boris\PYZus{}stepT} \PY{o}{*} \PY{n}{c2Boris\PYZus{}vnew\PYZus{}p7}\PY{p}{)}
\PY{n}{np}\PY{o}{.}\PY{n}{isclose}\PY{p}{(}\PY{n}{c2Boris\PYZus{}pnew\PYZus{}p7}\PY{p}{,} \PY{n}{c2Boris\PYZus{}p734\PYZus{}p}\PY{p}{[}\PY{l+m+mi}{1}\PY{p}{]}\PY{p}{)}
\end{Verbatim}
\end{tcolorbox}

            \begin{tcolorbox}[breakable, size=fbox, boxrule=.5pt, pad at break*=1mm, opacityfill=0]
\prompt{Out}{outcolor}{36}{\boxspacing}
\begin{Verbatim}[commandchars=\\\{\}]
array([ True,  True,  True])
\end{Verbatim}
\end{tcolorbox}
        
    \begin{tcolorbox}[breakable, size=fbox, boxrule=1pt, pad at break*=1mm,colback=cellbackground, colframe=cellborder]
\prompt{In}{incolor}{37}{\boxspacing}
\begin{Verbatim}[commandchars=\\\{\}]
\PY{n}{c2Boris\PYZus{}pnew\PYZus{}p7}
\end{Verbatim}
\end{tcolorbox}

            \begin{tcolorbox}[breakable, size=fbox, boxrule=.5pt, pad at break*=1mm, opacityfill=0]
\prompt{Out}{outcolor}{37}{\boxspacing}
\begin{Verbatim}[commandchars=\\\{\}]
array([-0.48495339, -0.00433543,  0.00569717])
\end{Verbatim}
\end{tcolorbox}
        
    \begin{tcolorbox}[breakable, size=fbox, boxrule=1pt, pad at break*=1mm,colback=cellbackground, colframe=cellborder]
\prompt{In}{incolor}{ }{\boxspacing}
\begin{Verbatim}[commandchars=\\\{\}]

\end{Verbatim}
\end{tcolorbox}


    % Add a bibliography block to the postdoc
    
    
    
\end{document}
