\documentclass[12pt]{article}

\usepackage[margin=1in]{geometry}
\usepackage{amsfonts,amsmath,amssymb}
\usepackage{multicol}

\usepackage{graphicx}
\usepackage{float}
\usepackage[nottoc, notlot, notlof]{tocbibind}
\usepackage{hyperref}
\usepackage{enumitem}
\usepackage{caption}
\usepackage{subcaption}
\usepackage[T1]{fontenc}
\usepackage[utf8]{inputenc}
\usepackage{xcolor}
\newcommand{\leavealine}
{\vskip 0.5cm}


% for equation box
\usepackage{color}
\definecolor{myblue}{rgb}{.8, .8, 1}

\usepackage[most]{tcolorbox}
\tcbset{
	enhanced,
	colback=myblue!100!white,
	boxrule=0.1pt,
	colframe=myblue!100!black,
	fonttitle=\bfseries
}
%until here

\begin{document}
	{\fontfamily{ppl}\selectfont 
		\begin{center}
			\Large{\textbf{Magnetic Mirror Effect in Magnetron Plasma:}} \\
			\Large{\textbf{Modeling of Plasma Parameters}} \\
		\end{center}
		%\color{blue}
		
		\section{Lorentz Force}
		The equations of motion for a charged particle under the influence of Electric and Magnetic fields is described by the Lorentz Force in the S.I. units as
		\begin{equation}
			\label{eqn:lorentz}
			\frac{d \textbf{v}}{d t} = \frac{q}{m} \left(\textbf{E} + \textbf{v} \times \textbf{B} \right)
		\end{equation}
		along with the expression for the velocity
		\begin{equation}
			\label{eqn:velocity}
			\frac{d \textbf{x}}{d t} = \textbf{v}
		\end{equation} 
	
		\noindent These equations are discretized to obtain
		\begin{equation}
			\label{eqn:Dlorentz}
			\frac{\textbf{v}_{k+1} - \textbf{v}_{k}}{\Delta t} = \frac{q}{m} \left[\textbf{E}_{k} + \frac{\left( \textbf{v}_{k+1} + \textbf{v}_{k} \right) }{2} \times \textbf{B}_{k} \right] 
		\end{equation}
		from the Lorentz Force equation ( \ref{eqn:lorentz} ), and
		\begin{equation}
			\label{eqn:Dvelocity}
			\frac{\textbf{x}_{k+1} - \textbf{x}_{k} }{\Delta t} =	\textbf{v}_{k+1}
		\end{equation}
		from the expression for velocity in equation ( \ref{eqn:velocity} ).

	
		\section{Boris Algorithm}
		The Boris Algorithm splits equation ( \ref{eqn:Dlorentz} ) into three equations.
		\begin{equation}
			\label{eqn:halfEpulse1}
			\frac{\textbf{v}^{-} - \textbf{v}_{k}}{\left( \Delta t  / 2 \right)} = \frac{q}{m} \textbf{E}_{k} \hspace{1cm} or \hspace{1cm}	
			\frac{\textbf{v}^{-} - \textbf{v}_{k}}{ \Delta t } = \frac{1}{2}\frac{q}{m} \textbf{E}_{k}
		\end{equation}
		which is often called the first half of the electric pulse.
		\begin{equation}
			\label{eqn:Brotation}
			\frac{\textbf{v}^{+} - \textbf{v}^{-}}{ \Delta t } = \frac{q}{m} \left(\frac{\textbf{v}^{+} + \textbf{v}^{-}}{ 2 }\right) \textbf{B}_{k}
		\end{equation}
		which is often called rotation by the magnetic field.
		\begin{equation}
			\label{eqn:halfEpulse2}
			\frac{\textbf{v}_{k+1} - \textbf{v}^{+}}{\left( \Delta t  / 2 \right)} = \frac{q}{m} \textbf{E}_{k} \hspace{1cm} or \hspace{1cm}	
			\frac{\textbf{v}_{k+1} - \textbf{v}^{+}}{ \Delta t } = \frac{1}{2}\frac{q}{m} \textbf{E}_{k}
		\end{equation}
		which is often called the second half of the electric pulse.
		
		\noindent Adding equations (\ref{eqn:halfEpulse1}), (\ref{eqn:Brotation}) and (\ref{eqn:halfEpulse2}) gives
		$$ \label{eqn:Dlorentz}
		\frac{\textbf{v}_{k+1} - \textbf{v}_{k}}{\Delta t} = \frac{q}{m} \left[\textbf{E}_{k} + \frac{\left( \textbf{v}^{+} + \textbf{v}^{-} \right) }{2} \times \textbf{B}_{k} \right] $$ which is almost the discretized Lorentz equation ( \ref{eqn:Dlorentz}) except that $\left( \textbf{v}^{+} + \textbf{v}^{-} \right) $ is substituted for $\left( \textbf{v}^{k+1} + \textbf{v}^{k}\right) $. However, subtracting equation (\ref{eqn:halfEpulse1}) from equation (\ref{eqn:halfEpulse2}) gives $\left( \textbf{v}^{+} + \textbf{v}^{-} \right) = \left( \textbf{v}^{k+1} + \textbf{v}^{k}\right) $, giving the discretized Lorentz equation ( \ref{eqn:Dlorentz} ). This means that the Boris algorithm is equivalent to the discretized Lorentz equation. \\
		
		\noindent
		The equations (\ref{eqn:halfEpulse1}), (\ref{eqn:Brotation}) and (\ref{eqn:halfEpulse2}) can be written slightly different as 
		 
		 \begin{center}
		 \begin{tcolorbox}[width=8cm]
		 
		 \begin{equation}
		 		\begin{split}
		 		\textbf{v}^{-} & = \textbf{v}_{k} + q^{\prime} \textbf{E}_{k} \\
		 		\textbf{v}^{+} & = \textbf{v}^{-} + 2 q^{\prime} \left( \textbf{v}^{-} \times \textbf{B}_{k} \right) \\
		 		\textbf{v}_{k+1} & = \textbf{v}^{+} + q^{\prime} \textbf{E}_{k} \\
		 		\textbf{x}_{k+1} & = \textbf{x}_{k} + \Delta t \hspace{0.2cm}\textbf{v}_{k+1}
		 		\end{split}	 			
	 	\end{equation}
		\end{tcolorbox}
		\end{center}
	
	alongwith equation (\ref{eqn:Dvelocity}) where $q^{\prime} = \frac{\displaystyle q}{\displaystyle m} \frac{\displaystyle \Delta t}{ 2}$. These equations are used to update the velocity of the particle under the influence of the Lorentz force under the Boris update strategy.
	
	\begin{thebibliography}{}
		\bibitem{Borisgood}
		Qin, H., Zhang, S., Xiao, J., $\&$ Tang, W. M. (April, 2013). \textit{Why is Boris algorithm so good?}. Princeton Plasma Physics Laboratory, PPPL-4872.
		
	\end{thebibliography}		
\end{document}