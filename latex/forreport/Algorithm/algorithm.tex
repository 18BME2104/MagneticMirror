\documentclass[12pt]{article}

\usepackage[margin=1in]{geometry}
\usepackage{amsfonts,amsmath,amssymb}
\usepackage{multicol}

\usepackage{graphicx}
\usepackage{float}
\usepackage[nottoc, notlot, notlof]{tocbibind}
\usepackage{hyperref}
\usepackage{enumitem}
\usepackage{caption}
\usepackage{subcaption}
\usepackage[T1]{fontenc}
\usepackage[utf8]{inputenc}
\usepackage{xcolor}
\newcommand{\leavealine}
{\vskip 0.5cm}


% for equation box
\usepackage{color}
\definecolor{myblue}{rgb}{.8, .8, 1}

\usepackage[most]{tcolorbox}
\tcbset{
	enhanced,
	colback=myblue!100!white,
	boxrule=0.1pt,
	colframe=myblue!100!black,
	fonttitle=\bfseries
}
%until here

% for python code
\usepackage[utf8]{inputenc}

% Default fixed font does not support bold face
\DeclareFixedFont{\ttb}{T1}{txtt}{bx}{n}{12} % for bold
\DeclareFixedFont{\ttm}{T1}{txtt}{m}{n}{12}  % for normal

% Custom colors
\definecolor{deepblue}{rgb}{0,0,0.5}
\definecolor{deepred}{rgb}{0.6,0,0}
\definecolor{deepgreen}{rgb}{0,0.5,0}

% \usepackage{listings}

% Python style for highlighting
\usepackage[utf8]{inputenc}

% Default fixed font does not support bold face
\DeclareFixedFont{\ttb}{T1}{txtt}{bx}{n}{12} % for bold
\DeclareFixedFont{\ttm}{T1}{txtt}{m}{n}{12}  % for normal

% Custom colors
\usepackage{color}
\definecolor{deepblue}{rgb}{0,0,0.5}
\definecolor{deepred}{rgb}{0.6,0,0}
\definecolor{deepgreen}{rgb}{0,0.5,0}
\definecolor{almostwhite}{rgb}{0.99,0.99, 0.99}

\usepackage{listings}

% Python style for highlighting
\lstset{
		basicstyle  =   \footnotesize,
		keywordstyle    = \color{deepred}\bfseries,
		stringstyle     = \color{strings},
		identifierstyle = \color{black},
		commentstyle    = \color{deepgreen},
		breaklines=true,
		numbersep=-10pt,
		stepnumber=1,
		showspaces=false,
		escapechar=§,
		showstringspaces=false,
		showtabs=false,
		frame=single,  
		rulecolor=\color{black},
		tabsize=4,              
		captionpos=t,           
		breaklines=true,
		breakatwhitespace=false,
		numbers=left,
		extendedchars=\true,
		emph=[3]{href, Particle, Boris_update, Field, uniform_E_field, radial_E_field, uniform_B_field, helmholtz_coil_B_field, two_helmholtz_B_field, Sampler, sample_uniform_position, sample_uniform_velocity, sample_velocity_uniformKE, sample_Maxwellian_velocity, sample_parabolic_velocity},
		emphstyle=[3]{\color{deepblue}},
		backgroundcolor=\color{almostwhite},
		language=Python 
}

%until here

\begin{document}
	
	{\fontfamily{ppl}\selectfont 
		\begin{center}
			\Large{\textbf{Magnetic Mirror Effect in Magnetron Plasma:}} \\
			\Large{\textbf{Modeling of Plasma Parameters}} \\
		\end{center}
		%\color{blue}
		
		Different aspects of the program are discussed based on the different notebooks (ipynb files) that describe them.
		
		\begin{center}
			\begin{tcolorbox}[width=15cm]
				GitHub repository:
				\vspace{0.2cm} 
			\phantom{a} \\
		\phantom{push this repository}
	\textbf{\textit{https://github.com/18BME2104/MagneticMirror}} \\
\phantom{a}
			\end{tcolorbox}
		\end{center}
		
		\begin{center}
			\begin{tcolorbox}[width=12cm]
				Required functionalities and Files:
				\vspace{0.2cm} 
				\begin{enumerate}
					\item Particle - \textbf{particle.ipynb}
					\item Electric and Magnetic fields - \textbf{field.ipynb}
					\item Particle initialization - \textbf{sampling.ipynb}
					\item Updating the particles - \textbf{step.ipynb}
					\item Batches of updates - \textbf{run.ipynb}
					\item Plotting - \textbf{plot.ipynb}
				\end{enumerate}
			\end{tcolorbox}
		\end{center}
	
		\section{Constants - constants.ipynb}
		The \textbf{constants.ipynb} notebook describes some constants useful in the program. Some useful constants are $e$ (electron charge) $m_{e}$ (electron mass), charges and masses of ions in the plasma, $a.m.u$ (atomic mass unit), $N_{A}$ (Avogadro's number), $\epsilon_{0}$ (permittivity of vacuum), $\mu_{0}$ (permeability of vacuum), $K$ or $k_{B}$ (Boltzmann's constant), etc.
		
		\lstinputlisting[language=Python, linerange={1-6}]{constants.py}
		
		\section{Particle - particle.ipynb}
		The \textbf{particle.ipynb} notebook describes the state of a particle; its position, velocity, mass, charge, name, and optionally acceleration (which is set to 0 as default if it is not required to track the acceleration of a particle) as of now. \\
		
		\noindent Currently the \textbf{Boris algorithm} is an update strategy defined to update the state of a particle. Other strategies could be defined in new functions in the class. However, Boris algorithm is good enough for us to get started.
		
		\lstinputlisting[language=Python]{particle.py}
		
		\section{Electric and Magnetic fields - field.ipynb}
		Electric and Magnetic field configurations described in \textbf{field.ipynb}. \\
		
		\noindent Currently \textbf{Uniform Electric field} and the \textbf{Radial Electric field} (the field depends on the particle's position) created by an electrode are available to set up electric fields. \textbf{Uniform Magnetic field} and Magnetic field created by a \textbf{Helmholtz coil} and that by two Helmholtz coils are available. \\
		
		\noindent Other Electric and Magnetic fields can be defined as functions in this class. These fields could be based on modeling of apparatus used to create electric or magnetic fields such as coils, or based on analytic expressions.
		
		
		\lstinputlisting[language=Python]{field.py}
		
		
		\section{Particle initialization - sampling.ipynb}
		The initial positions and velocities of the particles play an important role in the evolution of their state under the influence of electric and magnetic fields. The initial distribution of positions and velocities define where the particles start in the setup (or lab apparatus); for example where they are injected into a sputtering chamber through valves, and what velocities they start with; for example what potential they are accelerated through or what parameters were used for the pumps used to pump the particles in.\\
		
		\noindent The tracking of initial distribution is also important in determining how the magnetic mirror effect is observed in the plasma. The Sampler class samples initial positions and velocities for a given number of particles based on some available schemes. \\
		
		\noindent Currently positions can be sampled such that all particles start at the \textbf{same position} (like for example if injected through a port), \textbf{at the same distance} but randomly distributed in angular positions relative to some point(origin as of now). Velocities can be sampled such that particles have the \textbf{same speed}, \textbf{same velocity} or \textbf{Maxwellian distributed speeds} or \textbf{Maxwellian distributed velocities}.   
		
		\lstinputlisting[language=Python, linerange={1-68, 77-77}]{sampling.py}
		% assert keyword causes some error 
		% WARNING:
		% changing code in sampling.py requires chaging
		% line numbers here 
		
		\noindent The initial positions and velocities could be passed around in lists but to be able to reuse some generated samples, and avoid sampling all the time, it is convenient to write sampled positions and velocities to files (csv format seems convenient).
		\lstinputlisting[language=Python, linerange={1-3, 69-77}]{sampling.py}
		% assert keyword causes some error 
		% WARNING:
		% changing code in sampling.py requires chaging
		% line numbers here 
		
		\section{Updating the particles - step.ipynb}
		The Step class is concerned with \textbf{initializing particles}, \textbf{defining the fields}, and \textbf{updating the states of the particles} (positions and velocities) under the action of electric and magnetic fields.
		\lstinputlisting[language=Python, linerange={1-20, 27-27}]{step.py}
		
		\noindent To use sampled positions and velocities that have been saved in csv files, some reading functionality is useful to load these positions and velocities to be used during initialization.
		\lstinputlisting[language=Python, linerange={1-3, 21-27}]{step.py}
		
		\section{Batches of updates - run.ipynb}
		The Run class is concerned with running the steps defined by the Step class in step.ipynb. This includes creating batches of particles, updating them under the influence of electric and magnetic fields and removing them if they move out of the region of interest or for example are absorbed during a coating process. New batches of particles can be created to model the flow of particles as in a plasma chamber setup. Functionality to change the fields; for example changing the Voltages in electrodes or Currents in the coils, are also defined. This better models the control of plasma supply and electric and magnetic field control apparatus as in a laboratory.
		\lstinputlisting[language=Python]{run.py}
		
		\section{Plotting - plot.ipynb}
		The plot class is used to define functionalities to draw particle positions, plot positions and velocities of particles, as the system evolves like a lab apparatus. These functionalities are imported into run.ipynb; and if required other notebooks, to be used when the simulation is running.
			
	\begin{thebibliography}{}
		\bibitem{Borisgood}
		Qin, H., Zhang, S., Xiao, J., $\&$ Tang, W. M. (April, 2013). \textit{Why is Boris algorithm so good?}. Princeton Plasma Physics Laboratory, PPPL-4872.
		
	\end{thebibliography}		
\end{document}