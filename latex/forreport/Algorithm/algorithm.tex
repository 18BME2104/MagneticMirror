\documentclass[12pt]{article}

\usepackage[margin=1in]{geometry}
\usepackage{amsfonts,amsmath,amssymb}
\usepackage{multicol}

\usepackage{graphicx}
\usepackage{float}
\usepackage[nottoc, notlot, notlof]{tocbibind}
\usepackage{hyperref}
\usepackage{enumitem}
\usepackage{caption}
\usepackage{subcaption}
\usepackage[T1]{fontenc}
\usepackage[utf8]{inputenc}
\usepackage{xcolor}
\newcommand{\leavealine}
{\vskip 0.5cm}


% for equation box
\usepackage{color}
\definecolor{myblue}{rgb}{.8, .8, 1}

\usepackage[most]{tcolorbox}
\tcbset{
	enhanced,
	colback=myblue!100!white,
	boxrule=0.1pt,
	colframe=myblue!100!black,
	fonttitle=\bfseries
}
%until here

% for python code
\usepackage[utf8]{inputenc}

% Default fixed font does not support bold face
\DeclareFixedFont{\ttb}{T1}{txtt}{bx}{n}{12} % for bold
\DeclareFixedFont{\ttm}{T1}{txtt}{m}{n}{12}  % for normal

% Custom colors
\definecolor{deepblue}{rgb}{0,0,0.5}
\definecolor{deepred}{rgb}{0.6,0,0}
\definecolor{deepgreen}{rgb}{0,0.5,0}

% \usepackage{listings}

% Python style for highlighting
\usepackage[utf8]{inputenc}

% Default fixed font does not support bold face
\DeclareFixedFont{\ttb}{T1}{txtt}{bx}{n}{12} % for bold
\DeclareFixedFont{\ttm}{T1}{txtt}{m}{n}{12}  % for normal

% Custom colors
\usepackage{color}
\definecolor{deepblue}{rgb}{0,0,0.5}
\definecolor{deepred}{rgb}{0.6,0,0}
\definecolor{deepgreen}{rgb}{0,0.5,0}
\definecolor{almostwhite}{rgb}{0.99,0.99, 0.99}

\usepackage{listings}

% Python style for highlighting
\lstset{
		basicstyle  =   \footnotesize,
		keywordstyle    = \color{deepred}\bfseries,
		stringstyle     = \color{strings},
		identifierstyle = \color{black},
		commentstyle    = \color{deepgreen},
		breaklines=true,
		numbersep=-10pt,
		stepnumber=1,
		showspaces=false,
		escapechar=§,
		showstringspaces=false,
		showtabs=false,
		frame=single,  
		rulecolor=\color{black},
		tabsize=4,              
		captionpos=t,           
		breaklines=true,
		breakatwhitespace=false,
		numbers=left,
		extendedchars=\true,
		emph=[3]{href, Particle, Boris_update, Field, uniform_E_field, radial_E_field, uniform_B_field, helmholtz_coil_B_field, two_helmholtz_B_field, Sampler, sample_uniform_position, sample_uniform_velocity, sample_velocity_uniformKE, sample_Maxwellian_velocity, sample_parabolic_velocity},
		emphstyle=[3]{\color{deepblue}},
		backgroundcolor=\color{almostwhite},
		language=Python 
}

%until here

\begin{document}
	
	{\fontfamily{ppl}\selectfont 
		\begin{center}
			\Large{\textbf{Magnetic Mirror Effect in Magnetron Plasma:}} \\
			\Large{\textbf{Modeling of Plasma Parameters}} \\
		\end{center}
		%\color{blue}
		
		\section{Control on Particle Update strategies - NOT a control knob}
		Not really a control knob. Concerns precision of solution / update. \\
		
		\noindent Update strategies used in \textbf{particle.ipynb}\\
		
		\noindent Describe the Boris algorithm update strategy.
		
		\lstinputlisting[language=Python]{particle.py}
		
		\section{Control of Electric and Magnetic fields - Control Knobs here}
		Electric and Magnetic field configurations described in \textbf{field.ipynb} \\
		
		\lstinputlisting[language=Python]{field.py}
		
		\noindent Describe the Helmholtz coil magnetic field and electrode potential electric field configurations used. \\
		
		\noindent Different Electric field configurations could be used. Simple example: changing the electrode voltages. \\
		
		\noindent Different Magnetic field configurations could be used. Simple example: using many Helmholtz coils (number controllable), at different angles (angle controllable).
		
		
		\section{Controlling particle initialization - Control knobs here}
		Sampling particles with different initial velocities, and positions for example using different density functions f. For example: based on parameters like plasma Temperature. \\
		
		\noindent Different particle sampling and initialization strategies used in \textbf{sampling.ipynb} \\
		
		\lstinputlisting[language=Python]{sampling.py}
		
		\noindent Also track how the velocity distribution changes with time.
			
	\begin{thebibliography}{}
		\bibitem{Borisgood}
		Qin, H., Zhang, S., Xiao, J., $\&$ Tang, W. M. (April, 2013). \textit{Why is Boris algorithm so good?}. Princeton Plasma Physics Laboratory, PPPL-4872.
		
	\end{thebibliography}		
\end{document}