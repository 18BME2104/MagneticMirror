\documentclass[12pt]{article}

\usepackage[margin=1in]{geometry}
\usepackage{amsfonts,amsmath,amssymb}
\usepackage{multicol}

\usepackage{graphicx}
\usepackage{float}
\usepackage[nottoc, notlot, notlof]{tocbibind}
\usepackage{hyperref}
\usepackage{enumitem}
\usepackage{caption}
\usepackage{subcaption}
\usepackage[T1]{fontenc}
\usepackage[utf8]{inputenc}
\usepackage{xcolor}
\usepackage{pgfgantt}
\usepackage{rotating}
\usepackage[graphicx]{realboxes}

\usepackage{color}
\definecolor{myblue}{rgb}{.8, .8, 1}
\definecolor{LightGray}{gray}{0.9}

\usepackage[most]{tcolorbox}
\tcbset{
	enhanced,
	colback=myblue!100!white,
	boxrule=0.1pt,
	colframe=myblue!100!black,
	fonttitle=\bfseries
}

\usepackage{minted} % For code 
\usepackage{tikz} % For checkmark

\def\checkmark{\tikz\fill[scale=0.5](0,.35) -- (.25,0) -- (1,.7) -- (.25,.15) -- cycle;} %define checkmark

\begin{document}

	\fontfamily{ppl}\selectfont 
	\begin{center}
		\Large{\textbf{Magnetic Mirror Effect in Magnetron Plasma:}} \\
		\Large{\textbf{Modeling of Plasma Parameters}} \\
	\end{center}
	
	\section{Checks}
	To verify that our program is working correctly, we perform some checks. We compare results from the program with calculations done by hand; of certain quantities and see if they agree.
	
	\subsection{Sampling}
	As of now we our sampling of particle speeds is based on the Maxwell-Boltzmann distribution, that we have defined previously. We now check if the average speeds of the particles agrees with the calculations done by hand based on the plasma temperature.
	We recall the Maxwellian density function that we defined earlier. 
	\begin{equation}
		\label{eqn:maxwellian}
		\widehat{f_{M}} := \hat{f}(\boldsymbol{x}, \boldsymbol{v}, t) = \left(\frac{m}{2\pi KT}\right)^{\frac{3}{2}} \mathrm{exp}\left(-\frac{\boldsymbol{v}^{2}}{v_{th}^{2}}\right)
	\end{equation} where $$v_{th}^{2} = \frac{2 K T}{m}$$
	For our convenience, we use the density function with speed instead of velocity which is defined as:
	\begin{equation}
		\label{eqn:maxwellianSpeed}
		\widehat{f_{m}} := \hat{f}(\boldsymbol{x}, v, t) = \left(\frac{m}{2\pi KT}\right)^{\frac{3}{2}} 4 \pi v^{2} \: \mathrm{exp}\left(-\frac{v^{2}}{v_{th}^{2}}\right)
	\end{equation} which is obtained by integrating over the solid angle in the velocity variable. \\

	For this density function we calculate the average speed by with the expression:
	$$ \int_{v = - \infty}^{v = \infty} d v \: v \: \widehat{f_{m}} = \int_{v = - \infty}^{v = \infty} d v \: v \: \left(\frac{m}{2\pi KT}\right)^{\frac{3}{2}} 4 \pi v^{2} \: \mathrm{exp}\left(-\frac{v^{2}}{v_{th}^{2}}\right)$$
	$$=  4 \pi \left(\frac{m}{2\pi KT}\right)^{\frac{3}{2}} \int_{v = - \infty}^{v = \infty} d v \: v^{3} \: \mathrm{exp}\left(-\frac{v^{2}}{v_{th}^{2}}\right)$$ $$ = 4 \pi \left(\frac{m}{2\pi KT}\right)^{\frac{3}{2}} \frac{v_{th}^{4}}{2} = 4 \pi \left(\frac{1}{\pi v_{th}^{2}}\right)^{\frac{3}{2}} \frac{v_{th}^{4}}{2}$$ $$= \frac{2}{\sqrt{\pi}} v_{th} = \frac{2}{\sqrt{\pi}} \sqrt{\frac{2 K T}{m}} = \frac{2}{\sqrt{\pi}} \sqrt{\frac{2 R T}{M}} = $$ 
	\color{red}    \textbf{Task 1} \\
	do the calculation here ... \color{black}
	
	From section 1.1.1 Maxwellian sampling in the \textbf{studies.ipynb} notebook, we take the following:
	
	\begin{center}
		\begin{tcolorbox}[width=3.5cm]
			Initialization
		\end{tcolorbox}
	\end{center}
	\begin{minted}[
		frame=lines,
		framesep=2mm,
		baselinestretch=1.2,
		bgcolor=LightGray,
		fontsize=\footnotesize,
		linenos=true,
		breaklines
		]
		{python}
# c1Maxwell means checking the Maxwellian sampling
c1Maxwell = Run()
# Create 100 particles based on the data available in the files
c1Maxwell.create_batch_with_file_initialization('H+', constants.constants['e'][0], constants.constants['m_H'][0] * constants.constants['amu'][0], 100, 100, 'H ions', r_index=0, v_index=1)	
	\end{minted}
	\begin{center}
		\begin{tcolorbox}[width=3cm]
			Inspection
		\end{tcolorbox}
	\end{center}
	\begin{minted}[
		frame=lines,
		framesep=2mm,
		baselinestretch=1.2,
		bgcolor=LightGray,
		fontsize=\footnotesize,
		linenos=true,
		breaklines
		]
		{python}
# Take the 0th batch of particles
c1Maxwell_batch = c1Maxwell.batches[0]['H ions']
# Take the initial positions and velocities of the particles
c1Maxwell_positions = []
c1Maxwell_velocities = []
for particle in c1Maxwell_batch.particles:
c1Maxwell_positions.append(particle.r)
c1Maxwell_velocities.append(particle.v)
# Let's now look at the velocities
c1Maxwell_velocities
# We need to check if they are really Maxwellian distributed
# Get the speeds 
c1Maxwell_speeds = np.sqrt( \
[ (c1Maxwell_velocities[i][0] ** 2) + (c1Maxwell_velocities[i][1] ** 2) + \
(c1Maxwell_velocities[i][2] ** 2) for i in range(len(c1Maxwell_velocities)) ] )
c1Maxwell_meanspeed = np.sum(c1Maxwell_speeds) / c1Maxwell_speeds.size	
	\end{minted}
We get an average speed of the distribution to be:
\begin{tcolorbox}
	\begin{verbatim}
	14202.764572898674
\end{verbatim}
\end{tcolorbox}

\color{red}  \textbf{Task 2} \\
Check if $\frac{2}{\sqrt{\pi}} \sqrt{\frac{2 K T}{m}} = \frac{2}{\sqrt{\pi}} \sqrt{\frac{2 R T}{M}} = $ is what works out based on what I have used for sampling ... \color{black}
	\subsection{Update}
		We recall the Boris Algorithm, that we use to update the particles in the plasma simulation.
		\begin{center}
			\begin{tcolorbox}[width=8cm]
				
				\begin{equation}
					\label{Boris Algorithm}
					\begin{split}
						\boldsymbol{v}^{-} & = \boldsymbol{v}_{k} + q^{\prime} \mathbf{E}_{k} \\
						\boldsymbol{v}^{+} & = \boldsymbol{v}^{-} + 2 q^{\prime} \left( \boldsymbol{v}^{-} \times \mathbf{B}_{k} \right) \\
						\boldsymbol{v}_{k+1} & = \boldsymbol{v}^{+} + q^{\prime} \mathbf{E}_{k} \\
						\boldsymbol{x}_{k+1} & = \boldsymbol{x}_{k} + \Delta t \hspace{0.2cm}\boldsymbol{v}_{k+1}
					\end{split}	 			
				\end{equation}
			\end{tcolorbox}
		\end{center}
		where $q^{\prime} = \frac{\displaystyle q}{\displaystyle m} \frac{\displaystyle \Delta t}{ 2}$.
		
		To verify that the simulation works as we expect it to, we perform a calculation and compare it to the output of an algorithm. From section 1.2.1 Boris Update in the \textbf{studies.ipynb} notebook, we take the following:
		\begin{center}
			\begin{tcolorbox}[width=3.5cm]
				Initialization
			\end{tcolorbox}
		\end{center}

		\begin{minted}[
			frame=lines,
			framesep=2mm,
			baselinestretch=1.2,
			bgcolor=LightGray,
			fontsize=\footnotesize,
			linenos=true,
			breaklines
			]
			{python}
# c2Boris means checking the Boris update
c2Boris = Run()
#Create 10 particles
c2Boris.create_batch_with_file_initialization('H+', constants.constants['e'][0], constants.constants['m_H'][0] * constants.constants['amu'][0], 100, 10, 'H ions', r_index=0, v_index=1)		
		\end{minted}
		\begin{center}
			\begin{tcolorbox}[width=4.5cm]
				Update input data
			\end{tcolorbox}
		\end{center}		

		\begin{minted}[
			frame=lines,
			framesep=2mm,
			baselinestretch=1.2,
			bgcolor=LightGray,
			fontsize=\footnotesize,
			linenos=true,
			breaklines
			]
			{python}
											
c2Boris_index_update = 0 # Update the first batch in this Run instance run_Boris_check
c2Boris_particle_track_indices = [i for i in range(10)] # Track all 10 particles
c2Boris_dT = 10**(-6) # 1 microseconds
c2Boris_stepT = 10**(-7) # 0.1 microseconds time step
c2Boris_E0 = 1000 # say 1000 Volts (voltage) per meter (size of chamber) 
c2Boris_Edirn = [1,0,0] #in the x-direction [1,0,0]
c2Boris_B0 = 10 * (10**(-3)) # Meant to say 10 mT 
c2Boris_Bdirn = [0,1,0] #in the y-direction [0,1,0]
c2Boris_argsE = [element * c2Boris_E0 for element in c2Boris_Edirn] # currently the uniform_E_field configuration is used
c2Boris_argsB = [element * c2Boris_B0 for element in c2Boris_Bdirn]# currently the uniform_B_field configuration is used
		\end{minted}
	
		\begin{center}
		\begin{tcolorbox}[width=2.5cm]
			Update
		\end{tcolorbox}
		\end{center}
	
	\begin{minted}[
		frame=lines,
		framesep=2mm,
		baselinestretch=1.2,
		bgcolor=LightGray,
		fontsize=\footnotesize,
		linenos=true,
		breaklines,
		]          
		{python}
c2Boris_positions_and_velocities = c2Boris.update_batch_with_unchanging_fields(c2Boris_index_update, c2Boris_dT, c2Boris_stepT, c2Boris_argsE, c2Boris_argsB, c2Boris_particle_track_indices)	

#Let's inspect the positions and velocities of the particles at index 1 and 7
c2Boris_p1 = c2Boris_positions_and_velocities[1]
c2Boris_p7 = c2Boris_positions_and_velocities[7]

#Let's take the positions and velocities of the particle 1 after the 3rd and the 4th time steps.
c2Boris_p134_p = [c2Boris_p1[3][1], c2Boris_p1[4][1]]
c2Boris_p134_v = [c2Boris_p1[3][2], c2Boris_p1[4][2]]

#Similary for particle 7
c2Boris_p734_p = [c2Boris_p7[3][1], c2Boris_p7[4][1]]
c2Boris_p734_v = [c2Boris_p7[3][2], c2Boris_p7[4][2]]	
	\end{minted}
The program gives the following output. For example, we can check the particles 1 and 7 in the update. After the $3^{rd}$ step of the update, the velocity of particle 1 was:
\begin{tcolorbox}
	\begin{verbatim}
		[39771.83801956555, -5029.491038, 4533.343932509505]
	\end{verbatim}
\end{tcolorbox}
and after the $4^{th}$ update, it changed to:
\begin{tcolorbox}
	\begin{verbatim}
	[48909.86581773698, -5029.491038, 8798.399243869582]
\end{verbatim}
\end{tcolorbox}
Its position after the $3^{rd}$ step of the update was:
\begin{tcolorbox}
	\begin{verbatim}
	[-0.48985112694809785, -0.0020117964152, 0.00017005183797547688]
\end{verbatim}
\end{tcolorbox}
which was updated after the $4^{th}$ step of the update to:
\begin{tcolorbox}
	\begin{verbatim}
[-0.4849601403663242, -0.0025147455189999997, 0.001049891762362435] 
\end{verbatim}
\end{tcolorbox}
We now check if the same if true, with a calculation of the update done by hand.
$$q^{\prime} = \frac{\displaystyle q}{\displaystyle m} \frac{\displaystyle \Delta t}{ 2} = \frac{\displaystyle 1.602176634 \times 10^{-19} \: \mathrm{C}} {\displaystyle 1.008 \: \mathrm{a.m.u.} \times 1.6605390666 \times 10^{-27} \: \mathrm{kg} \: \mathrm{a.m.u.}^{-1}} \frac{\displaystyle 10^{-7} \: \mathrm{s}}{ 2} = 4.78597877761177 \: \mathrm{C} \: \mathrm{s} \: \mathrm{kg}^{-1}$$

$\boldsymbol{v}^{-} = \boldsymbol{v}_{3} + q^{\prime} \mathbf{E} = \begin{bmatrix} 
	39771.83801956555 \\ -5029.491038 \\ 4533.343932509505
\end{bmatrix} + 4.78597877761177 \begin{bmatrix}
 1000 \\ 0 \\ 0 
\end{bmatrix} = \begin{bmatrix}
44557.81679718 \\ -5029.491038 \\  4533.34393251
\end{bmatrix}$
$$\boldsymbol{v}^{+} = \boldsymbol{v}^{-} + 2 q^{\prime} \left( \boldsymbol{v}^{-} \times \mathbf{B} \right) = \begin{bmatrix}
	44557.81679718 \\ -5029.491038 \\  4533.34393251
\end{bmatrix} + 2 \cdot 4.78597877761177 \left( \begin{bmatrix}
44557.81679718 \\ -5029.491038 \\  4533.34393251
\end{bmatrix} \times \begin{bmatrix}
0.0 \\ 0.01 \\ 0.0
\end{bmatrix} \right)$$ $$ = \begin{bmatrix}
44123.88704013 \\ -5029.491038 \\ 8798.39924387
\end{bmatrix}$$
\begin{tcolorbox}
	$$\boldsymbol{v}_{4} = \boldsymbol{v}^{+} + q^{\prime} \mathbf{E} = \begin{bmatrix}
	44123.88704013 \\ -5029.491038 \\ 8798.39924387
\end{bmatrix} + 4.78597877761177 \begin{bmatrix}
1000 \\ 0 \\ 0 
\end{bmatrix} = \begin{bmatrix}
48909.86581774 \\ -5029.491038 \\  8798.39924387
\end{bmatrix}$$
\end{tcolorbox}
\begin{tcolorbox}
	$$\boldsymbol{x}_{4} = \boldsymbol{x}_{3} + \Delta t \hspace{0.2cm}\boldsymbol{v}_{4} = \begin{bmatrix}
	-4.89851127 \times 10^{-01} \\ -2.01179642\times 10^{-03} \\  1.70051838\times 10^{-04}
\end{bmatrix} + 10^{-7} \hspace{0.2cm}\ \begin{bmatrix}
48909.86581774 \\ -5029.491038 \\  8798.39924387
\end{bmatrix} = \begin{bmatrix}
-0.48496014 \\ -0.00251475 \\  0.00104989
\end{bmatrix}$$
\end{tcolorbox}
We see that the values for $\boldsymbol{x_{3}}$, $\boldsymbol{x_{4}}$, $\boldsymbol{v_{3}}$ and $\boldsymbol{v_{4}}$ for particle 1 are close (the latter digits differ due to the differing precision of calculations); i.e. the same when done by hand and in the program.

\noindent Similarly, for particle 7. After the $3^{rd}$ step of the update, the velocity of the particle was:
\begin{tcolorbox}
	\begin{verbatim}
[38895.85871094631, -8670.854777, 13914.969423620274]
\end{verbatim}
\end{tcolorbox}
and after the $4^{th}$ update, it changed to:
\begin{tcolorbox}
	\begin{verbatim}
[47135.881299118584, -8670.854777, 18096.176367366777]
\end{verbatim}
\end{tcolorbox}
Its position after the $3^{rd}$ step of the update was:
\begin{tcolorbox}
	\begin{verbatim}
[-0.4896669752432719, -0.0034683419108, 0.0038875496903932644] 
\end{verbatim}
\end{tcolorbox}
which was updated after the $4^{th}$ step of the update to:
\begin{tcolorbox}
	\begin{verbatim}
[-0.48495338711336006, -0.0043354273885, 0.0056971673271299424]
\end{verbatim}
\end{tcolorbox}

$\boldsymbol{v}^{-} = \boldsymbol{v}_{3} + q^{\prime} \mathbf{E} = \begin{bmatrix} 
	 38895.85871095 \\ -8670.854777 \\ 13914.96942362
\end{bmatrix} + 4.78597877761177 \begin{bmatrix}
	1000 \\ 0 \\ 0 
\end{bmatrix} = \begin{bmatrix}
	43681.83748856 \\ -8670.854777 \\ 13914.96942362
\end{bmatrix}$
$$\boldsymbol{v}^{+} = \boldsymbol{v}^{-} + 2 q^{\prime} \left( \boldsymbol{v}^{-} \times \mathbf{B} \right) = \begin{bmatrix}
	43681.83748856 \\ -8670.854777 \\ 13914.96942362
\end{bmatrix} + 2 \cdot 4.78597877761177 \left( \begin{bmatrix}
	43681.83748856 \\ -8670.854777 \\ 13914.96942362
\end{bmatrix} \times \begin{bmatrix}
	0.0 \\ 0.01 \\ 0.0
\end{bmatrix} \right)$$ $$ = \begin{bmatrix}
	42349.90252151 \\ -8670.854777  \\ 18096.17636737
\end{bmatrix}$$
\begin{tcolorbox}
	$$\boldsymbol{v}_{4} = \boldsymbol{v}^{+} + q^{\prime} \mathbf{E} = \begin{bmatrix}
	42349.90252151 \\ -8670.854777  \\ 18096.17636737
\end{bmatrix} + 4.78597877761177 \begin{bmatrix}
	1000 \\ 0 \\ 0 
\end{bmatrix} = \begin{bmatrix}
	47135.88129912 \\ -8670.854777 \\ 18096.17636737
\end{bmatrix}$$
\end{tcolorbox}
\begin{tcolorbox}
	$$\boldsymbol{x}_{4} = \boldsymbol{x}_{3} + \Delta t \hspace{0.2cm}\boldsymbol{v}_{4} = \begin{bmatrix}
	-0.48966698 \\ -0.00346834 \\  0.00388755
\end{bmatrix} + 10^{-7} \hspace{0.2cm}\ \begin{bmatrix}
	47135.88129912 \\ -8670.854777 \\ 18096.17636737
\end{bmatrix} = \begin{bmatrix}
	-0.48495339 \\ -0.00433543 \\  0.00569717
\end{bmatrix}$$
\end{tcolorbox}
We see that the values for $\boldsymbol{x_{3}}$, $\boldsymbol{x_{4}}$, $\boldsymbol{v_{3}}$ and $\boldsymbol{v_{4}}$ for particle 7 are close (the latter digits differ due to the differing precision of calculations); i.e. the same when done by hand and in the program.	

Based on these calculations, we believe that our particle update works as expected; according to the Boris Algorithm, and this part of the program works correctly. \color{green} \checkmark \color{black}  
		
	\section{Plasma Stream}
	\color{red}
	\begin{itemize}
		\item Maxwellian distribution
		\item maybe Parabolic distribution
	\end{itemize}
	\begin{itemize}
		\item Change Voltage of electrode to change electric field
		\item Change current in Helmholtz coil to change the magnetic field 
	\end{itemize}
		while the updates are running.
		on different batches.
	\color{black}	
	
	\section{The Magnetic Mirror configuration}
	\color{red}
	\textbf{need parabolic $\mathbf{B}$ along $z$-axis configuration for $\nabla$$\mathbf{B}$ $\parallel$ $\mathbf{B}$ }
	\color{black}
\end{document}